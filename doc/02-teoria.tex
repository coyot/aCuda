
\chapter{Podstawy teoretyczne}

W rozdziale tym przedstawiony zostanie przegląd literatury, który stanowi podstawy wiedzy na temat eksploracji danych, a w szczególności problemu odkrywania reguł asocjacyjnych w dużych zbiorach danych.

\section{Definicje}

\subsection{Model teoretyczny}

Niech $\mathbf{I} = \mathbf{I}_1, \mathbf{I}_2, \ldots, \mathbf{I}_m$ będzie zbiorem binarnych atrybutów, zwanych elementami lub towarami. Niech $\mathbf{T}$ będzie zbiorem transakcji. Każda z transakcji $t$ jest reprezentowana, jako wektor wartości binarnych, gdzie $t[k]=1$ oznacza, że w transakcji $t$ zakupiony został element $\mathbf{I}_k$, w przeciwny wypadku $t[k]=0$. W bazie danych znajduje się jedna krotka odpowiadająca jednej transakcji. Niech $\mathbf{X} \subseteq \mathbf{I}$, wówczas transakcja $t$ spełnia $\mathbf{X}$, jeżeli dla wszystkich elementów $\mathbf{I}_k \in \mathbf{X}$, $t[k]=1$.

\begin{df}\label{regula:def}
Reguła asocjacyjna ma postać $\mathbf{X} \Rightarrow \mathbf{Y}$, gdzie $\mathbf{X}, \mathbf{Y} \subseteq \mathbf{I}$ oraz $\mathbf{X} \cap \mathbf{Y} = \oslash$.
\end{df}

Reguła asocjacyjna zdefiniowana w Definicji~\ref{regula:def} określa, że jeśli dany klient kupił towary ze zbioru $\mathbf{X}$, najprawdopodobniej kupi też towary ze zbioru $\mathbf{Y}$. Aby reguła asocjacyjna stanowiła interesujące źródło informacji dla analityka stosującego techniki eksploracji danych, musi ona spełniać określone warunki wyrażone za pomocą odpowiednich miar. Dwie najbardziej popularne miary jakości reguł asocjacyjnych to poziom pokrycia (\\english{support}) oraz poziom ufności (\english{confidence}).

\begin{df}\label{support:def}
Poziom pokrycia dla reguły asocjacyjnej postaci $\mathbf{X} \Rightarrow \mathbf{Y}$ jest miarą definiowaną dla zbioru $\mathbf{Z} = \mathbf{X} \cup \mathbf{Y}$ - określa ona częstotliwość wystąpień zbioru $\mathbf{Z}$ w zbiorze transakcji $\mathbf{T}$.
\end{df}

Oznacza to, że poziom pokrycia ($sup$) jest stosunkiem liczby transakcji zawierających elementy sumy zbiorów (czyli $|\mathbf{Z}|$, gdzie $\mathbf{Z} = \mathbf{X} \cup \mathbf{Y}$) do liczby wszystkich transakcji w systemie ($|\mathbf{T}|$). Wzór \ref{support2:def} przedstawia sposó obliczania wartości $sup$.

\begin{equation}\label{support2:def}
	sup = \frac{|\mathbf{Z}|}{|\mathbf{T}|}
\end{equation}

Łatwo zauważyć, że jeśli poziom ten jest niski, to oznacza to, że nie ma jednoznacznych dowodów na łączne występowanie elementów zbioru $\mathbf{Z} = \mathbf{X} \cup \mathbf{Y}$, ponieważ zbiór $\mathbf{Z}$ występuje w niewielkiej liczbie transakcji. 

Poziom ufności jest miarą zdefiniowaną dla implikacji reprezentowanej przez regułę asocjacyjną~\cite{Elmasri:db}. 

\begin{df}\label{confidence:def}
Pioziom ufności reguły asocjacyjnej $\mathbf{X} \Rightarrow \mathbf{Y}$ jest równy 
\begin{equation}
	conf = \frac{sup(\mathbf{Z})}{sup(\mathbf{X})}
\end{equation}

gdzie $\mathbf{Z} = \mathbf{X} \cup \mathbf{Y}$.
\end{df}

Ze wzoru wynika, że poziom ufności jest prawdopodobieństwem tego, że elementy tworzące zbiór $\mathbf{Y}$ zostaną kupione przez danego klienta pod warunkiem, że klient kupi elementy należące do zbioru $\mathbf{X}$.

Ważnym jest by nie mylić poziomu ufności z poziomem pokrycia. Podczas gdy ufność określa ''siłę'' reguły, pokrycie podkreśla jej statystyczną ważność. Inną motywacją, poza statystyczną ważnością, dlaczego interesujący jest ten współczynnik, jest fakt, że poszukiwane reguły powinny spełniać pewne wymaganie co do wysokości wartości $sup$ z powodów biznesowych. Jeśli poziom pokrycia nie jest wystarczająco wysoki, to oznacza to, iż reguła nie jest warta brania pod uwagę, bądź jest ona mniej preferowana (może być rozpatrzona później)~\cite{Problem:Statement}.