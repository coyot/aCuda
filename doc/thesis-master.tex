
% Szkielet dla pracy pisanej w języku polskim.

\documentclass[polish,a4paper,oneside]{ppfcmthesis}
\usepackage[utf8]{inputenc}
\usepackage[OT4]{fontenc}
\usepackage{amsfonts}
\usepackage{listings}
\usepackage[mathcal]{eucal}
\usepackage{clrscode}
\usepackage{amsthm}
\usepackage{amsmath}
\usepackage{here}

\authortitle{inż.~}                                        % You can place "inż.~" here, if you really want to.
\author{Tomasz Kujawa}                              % Your name comes here
\title{Równoległe odkrywanie reguł asocjacyjnyh zaimplementowane na procesory graficzne}                   % Note how we protect the final title phrase from breaking
\ppsupervisor{dr ~inż.~Witold Andrzejwski} % Your supervisor comes here.
\ppyear{2011}                                         % Year of final submission (not graduation!)

\theoremstyle{definition} 
\newtheorem{twr}{Twierdzenie}
\newtheorem{lem}[twr]{Lemat}
\newtheorem{df}{Definicja}
\newtheorem{alg}{Algorytm}
\newcommand{\CommentSymbol}{\texttt{\textbf{/\hspace*{-0.3em}/}}}

\lstdefinelanguage{CSharp}
{
 morecomment = [l]{//}, 
 morecomment = [l]{///},
 morecomment = [s]{/*}{*/},
 morestring=[b]", 
 sensitive = true,
 morekeywords = {abstract,  event,  new,  struct,
   as,  explicit,  null,  switch,
   base,  extern,  object,  this,
   bool,  false,  operator,  throw,
   break,  finally,  out,  true,
   byte,  fixed,  override,  try,
   case,  float,  params,  typeof,
   catch,  for,  private,  uint,
   char,  foreach,  protected,  ulong,
   checked,  goto,  public,  unchecked,
   class,  if,  readonly,  unsafe,
   const,  implicit,  ref,  ushort,
   continue,  in,  return,  using,
   decimal,  int,  sbyte,  virtual,
   default,  interface,  sealed,  volatile,
   delegate,  internal,  short,  void,
   do,  is,  sizeof,  while,
   double,  lock,  stackalloc,   
   else,  long,  static,   
   enum,  namespace,  string, var, from, select, where}
}

\lstset{ %
language=CSharp,                % the language of the code
basicstyle=\footnotesize,       % the size of the fonts that are used for the code
numbers=left,                   % where to put the line-numbers
numberstyle=\footnotesize,      % the size of the fonts that are used for the line-numbers
stepnumber=2,                   % the step between two line-numbers. If it's 1, each line 
                                % will be numbered
numbersep=5pt,                  % how far the line-numbers are from the code
backgroundcolor=\color{white},  % choose the background color. You must add \usepackage{color}
showspaces=false,               % show spaces adding particular underscores
showstringspaces=false,         % underline spaces within strings
showtabs=false,                 % show tabs within strings adding particular underscores
frame=single,                   % adds a frame around the code
tabsize=2,                      % sets default tabsize to 2 spaces
captionpos=b,                   % sets the caption-position to bottom
breaklines=true,                % sets automatic line breaking
breakatwhitespace=false,        % sets if automatic breaks should only happen at whitespace
title=\lstname,                 % show the filename of files included with \lstinputlisting;
                                % also try caption instead of title
escapeinside={\%*}{*)},         % if you want to add a comment within your code
morekeywords={*,...}            % if you want to add more keywords to the set
}

\begin{document}

% Front matter starts here
\frontmatter\pagestyle{empty}%
\maketitle\cleardoublepage%

% Blank info page for "karta dyplomowa"
\thispagestyle{empty}\vspace*{\fill}%
\begin{center}Tutaj przychodzi karta pracy dyplomowej;\\oryginał wstawiamy do wersji dla archiwum PP, w pozostałych kopiach wstawiamy ksero.\end{center}%
\vfill\cleardoublepage%

% Table of contents.
\pagenumbering{Roman}\pagestyle{ppfcmthesis}%
\tableofcontents* \cleardoublepage%

% Main content of your thesis starts here.
\mainmatter%
\chapter{Wstęp}
Proces informatyzacji przedsiębiorstw, rozpoczęty kilka dekad temu, wprowadził światową gospodarkę na nowe, dotąd nieznane tory rozwoju. Skrócenie procesu produkcyjnego, wprowadzenie kontroli komputerowych, czy też skomputeryzowanych maszyn skróciło i ułatwiło produkcję,~a~także zarządzanie procesami w firmach i przedsiębiorstwach. Przed ludźmi stanęły możliwości, ale także wyzwania, z~którymi nigdy wcześniej nikt nie musiał sobie radzić. Zmiany, jakie nastąpiły przez ostatnie trzy dekady są nieodwracalne i zmuszają programistów do tworzenia nowych aplikacji, które będą w stanie sprostać stawianym im wymaganiom.

Informatyzacja firm, instytucji oraz innych jednostek organizacyjnych powinna realizować dwa podstawowe cele. Z~jednej strony powinna ona usprawniać pracę pojedynczego pracownika poprzez automatyzację realizowanych przez niego rutynowych zadań. Dzięki wykorzystaniu możliwości komputerów działania te powinny być wykonywane szybciej i w sposób bardziej niezawodny. Z~drugiej strony celem informatyzacji jest wpływanie na działanie całych firm w wyniku wspomagania decyzji kadry zarządzającej przedsiębiorstwami. Szybka analiza bazująca na pełnej i aktualnej informacji o stanie firmy może ułatwić kadrze zarządzającej podejmowanie trafnych i szybkich decyzji o strategicznym znaczeniu dla rozwoju danego przedsiębiorstwa.

Wprowadzenie komputerów do właściwie każdej przestrzeni ludzkiego życia wpłynęło na wyprodukowanie olbrzymich ilości danych. Reprezentowane są one w sposób umożliwiający ich składowanie i przetwarzanie komputerowe przez aplikacje analityczne. W chwili obecnej ludzkość jest świadkiem eksplozji w produkcji danych produkowanych przez różnego rodzaju systemy komputerowe. Analiza tych danych przynieść może wymierne korzyści nie tylko w kwestiach finansowych, ale~również poznawczych. Dzięki analizie zebranych w przeszłości informacji możliwe jest lepsze dopasowanie planów w przyszłości - na tej podstawie planowane mogą być np. akcje marketingowe, czy~też~promocje w supermarketach spożywczych. Wykorzystanie wiedzy uzyskanej w ten sposób jest niezwykle szerokie i może być użyte w każdym obszarze działalności firmy.

Odkrycie zależności pomiędzy zgromadzonymi danymi bez zastosowania narzędzi informatycznych jest procesem bardzo skomplikowanym i wymagającym do realizacji dużo czasu. Przy obecnej złożoności większości systemów oraz rozmiarom danych produkowanych przez te systemy, koszt czasowy jest na tyle duży, że ręczna analiza tych danych stała się niemożliwa. Dlatego też tworzone są narzędzia umożliwiające odkrywanie prawidłowości w dużych zbiorach danych, by człowiek na tej podstawie mógł podejmować decyzje i wyciągać wnioski.

Dział informatyki, który zajmuje się odkrywaniem ukrytych dla człowieka prawidłowości i reguł w danych nazywa się eksploracją danych (\english{Data Mining}, w literaturze spotkać można również określenie drążenie danych, ekstrakcja danych, pozyskiwanie wiedzy, czy też wydobywanie danych~\cite{Elmasri:db}), który jest jednym z etapów procesu \termdef{odkrywania wiedzy z baz danych} (\acronym{KDD}, \english{Knowledge Discovery in Databases}). Proces odkrywania wiedzy w bazach danych obejmują zwykle działania bardziej złożone niż tylko eksploracja danych. Są to między innymi selekcja danych, transformacja lub kodowanie danych, czy też raportowanie i prezentowanie odkrytych informacji~\cite{Elmasri:db}. Eksploracja danych to proces odkrywania wiedzy w postaci nowych, użytecznych, poprawnych i zrozumiałych wzorców w bardzo dużych wolumenach danych~\cite{DataMiningStart}. Możliwości stosowania technik eksploracji danych w praktyce, wymagają efektywnych metod przeszukiwania ogromnych plików lub baz danych. Warto przy tym wspomnieć, że tego typu technologie nie są w chwili obecnej dobrze zintegrowane z systemami zarządzania bazami danych.

Eksploracja danych odbywa się najczęściej w środowisku baz lub hurtowni danych, które stanowią doskonałe źródła danych do analizy - głównie ze względu na łatwość dostępu oraz usystematyzowaną strukturę przechowywanych informacji. Ponieważ liczba odkrytych wzorców w wielu przypadkach może być bardzo duża, odkryte wzorce bardzo często zapisuje się w osobnych relacjach bazy lub hurtowni danych. Pozwala to na ich dalsze przetwarzanie w trybie off-line przez użytkowników końcowych. Pojęcie eksploracji zyskuje coraz większą popularność (również w wymiarze marketingowym) i jest wykorzystywane w wielu dziedzinach ludzkiego życia.

Jednym z najczęściej wykorzystywanych modeli wiedzy w eksploracji danych są reguły asocjacyjne. Reguła asocjacyjna ma postać $X \Rightarrow Y$, gdzie $X$ oraz $Y$ są wzajemnie rozłącznymi zbiorami elementów. Przykładem reguły, która mogła zostać odkryta w bazie danych sklepu komputerowego, może być reguła postaci $komputer \land myszka \Rightarrow monitor$. Prezentuje ona fakt, że klienci kupujący komputer oraz myszke z dużym prawdopodobieństwem kupią również monitor. W~\cite{Problem:Statement} po raz pierwszy sformułowany został problem odkrywania reguł asocjacyjnych wraz z algorytmem Apriori, który jest podstawą wielu algorytmów znajdujących reguły asocjacyjne. Algorytm ten został następnie rozszerzony w pracy~\cite{AssRulesStrt}.

W ostatnich latach pojawiły się nowe możliwości wykorzystania współczesnych komputerów. W roku 2007 firma NVIDIA udostępniła programistom uniwersalną architekturę obliczeniową CUDA (\english{Compute Unified Device Architecture}), który umożliwia wykorzystanie mocy obliczeniowej \termdef{procesorów graficznych} (\acronym{GPU}, \english{Graphics Processing Unit}), bądź innych procesorów wielordzeniowych, do rozwiązywania ogólnych problemów obliczeniowych w sposób znacząco wydajniejszy niż w przypadku tradycyjnych, sekwencyjnych procesorów~\cite{cuda:zone}. Choć w grach komputerowych moc obliczeniową jednostek graficznych można wykorzystać do obliczeń fizyki, to CUDA idzie jeszcze dalej, umożliwiając przyspieszenie obliczeń w takich dziedzinach, jak biologia, fizyka, kryptografia, bioinformatyka oraz innych naukach. Specjalnie dla potrzeb tego segmentu NVidia opracowała kartę graficzną o nazwie \emph{Tesla}~\cite{cuda:tesla}. Układy te są pierwszymi układami produkowanymi na masową skalę, które przeznaczone zostały do pracy \termdef{obliczeniach ogólnego przeznaczenia na układach GPU} (\acronym{GPGPU}, \english{General-Purpose Computing on Graphics Processing Units}), czyli segmencie do tej pory zarezerwowanym dla klasycznych procesorów obliczeniowych.

Do tej pory bardzo małe jest zainteresowanie wykorzystaniem tej~technologii w procesie odkrywania wiedzy, a~w~szczególności znajdowania reguł asocjacyjnych. Wyniki przeprowadzonych eksperymentów pozwalają przypuszczać, że algorytm wykorzystujący możliwości procesorów wielordzeniowych, a w szczególności GPU, będzie wyraźnie szybszy od klasycznych algorytmów eksploracji danych zaimplementowany na tradycyjnych procesorach.

\section{Cel i zakres pracy}
Celem pracy jest zaprojektowanie i zaimplementowanie algorytmu odkrywającego reguły asocjacyjne, który będzie wykorzystywał możliwośći współczesnych kart graficznych dzięki wykorzystaniu technologii CUDA oraz porównanie zaprojektowanego i zaimplementowanego algorytmu do innych, podstawowych algorytmów odkrywania reguł asocjacyjnych. W~ramach pracy dokonane zostanie również zebranie wiedzy dotyczącej algorytmów eksploracji reguł asocjacyjnych.

Rozdział 1 - wstęp.. Tutaj dalszy opis struktury pracy - zrobiony na koniec, gdy wszystko dalej będzie już znane.

\chapter{Podstawy teoretyczne}

W rozdziale tym przedstawiony zostanie przegląd literatury, który stanowi podstawy wiedzy na temat eksploracji danych, a w szczególności problemu odkrywania reguł asocjacyjnych w dużych zbiorach danych. Zebrana wiedza posłużyła autorowi do opracowania algorytmu wykorzystującego możliwości współczesnych kart graficznych.

\section{Definicje}

\subsection{Model teoretyczny}

Niech $I = \lbrace i_1,i_2,...,i_m \rbrace$ będzie \emph{zbiorem elementów} o liczności $|I| = m$. \emph{Transakcją} nazwano dowolny, niepusty podzbiór $X \subseteq I$ zbioru elementów. Bazą danych $DB$ nazwano dowolny zbiór par $(id, X)$, gdzie $X$ jest transakcją, a $id$ jest dowolną wartością unikalną w ramach bazy danych nazywaną \emph{identyfikatorem transakcji}. Bez utraty ogólności założono iż $id\in \mathbb{N}$. 

\emph{Wsparciem} (\english{support}) $sup(X)$ transakcji $X$, w bazie danych nazwano częstość wystąpień transakcji w bazie danych. Formalnie przedstawia to wzór~\ref{support:def}.

\begin{equation}
\label{support:def}
sup(X)=\frac{| \lbrace id: (id,Y)\in DB \wedge X\subseteq Y \rbrace |}{|DB|}
\end{equation} 

Łatwo zauważyć, że jeśli poziom ten jest niski, to oznacza to, że nie ma jednoznacznych dowodów na łączne występowanie elementów zbioru $Z = X \cup Y$, ponieważ zbiór $Z$ występuje w niewielkiej liczbie transakcji. 

\begin{df}\label{regula:def}
Niech będą dane dwie transakcje $X$ i $Y$ takie, że $X\cap Y=\emptyset$. Implikację postaci $X\Rightarrow  Y$ nazwano \emph{regułą asocjacyjną}.
\end{df}

\emph{Poziom ufności} (\english{confidence}) jest miarą zdefiniowaną dla implikacji reprezentowanej przez regułę asocjacyjną~\cite{Elmasri:db}. 

\begin{df}\label{confidence:def}
Pioziom ufności ($conf$) reguły asocjacyjnej $X \Rightarrow Y$ jest równy 
\begin{equation}
	conf(X \Rightarrow Y) = \frac{sup(X \cup Y)}{sup(X)}
\end{equation}
\end{df}

Po analizie definicji~\ref{confidence:def} łatwo zauważyć, że poziom ufności może być interpretowany, jako estymacja prawdopodobieństwa $P(Y | X)$.


Reguły asocjacyjne zazwyczaj powinny spełniać pewne wymagania zdefiniowane przez użytkownika - minimalne wsparcie oraz minimalny poziom ufności, oznaczane odpowiednio $minsup$ oraz $minconf$. Wyznaczają one dla aplikacji progi, jakie powinny spełniać zbiory oraz reguły, aby były brane pod uwagę w trakcie analizy.

\begin{df}
Zbiorem częstym $X \subseteq I$ nazywamy taki zbiór, który spełnia zależność $sup(X) \geq minsup$.
\end{df}

Generowanie reguł asocjacyjnych zazwyczaj sprowadza się do dwóch, niezależnych kroków:
\begin{enumerate}
	\item Minimalne wsparcie jest używane do odnalezienia wszystkich zbiorów częstych w bazie danych $DB$.
	\item Znalezione zbiory często oraz minimalny poziom ufności są używane do wygenerownia reguł asocjacyjnych.
\end{enumerate}

Znalezienie wszystkich zbiorów częstych w bazie danych $DB$ jest zadaniem wymagającym przeszukania wszystkich możliwych kombinacji bez powtórzeń ze zbioru $I$. Zbiór możliwych zbiorów elementów ma liczność równą $2^n - 1$ (wszystkie zbiory, poza zbiorem pustym, który nie jest w tym wypadku zbiorem sensownym w znaczeniu poddania go analizie). 

Warto zauważyć, że dla każdego zbioru częstego $Y$, każdy jego podzbiór $X$ jest również zbiorem częstym~\cite{Problem:Statement}. Korzystając z tej właściwości wsparcia możliwe jest w sposób efektywny znalezienie wszystkich zbiorów częstych w zadanej bazie danych - z tej zależności korzysta algorytm apriori opisany w rozdziale~\ref{apriori:section}. Dodatkowo, wszystkie reguły zbudowane na podstawie zbioru częstego $Y$ muszą spełniać warunek minimalnego wsparcia, ponieważ spełnia ten warunek zbiór $Y$, a suma zbiorów reguły jest zbiorem wyjściowym $Y$.

\section{Aktualna wiedza}
W rozdziale tym zebrana została oraz opracowana dotychczasowa wiedza (\english{state-of-the-art}) na temat odkrywania reguł asocjacyjnych. Przedstawione zostaną dwa podstawowe algorytmy wykorzystywane w tym procesie: Apriori oraz FP-growth. W chwili obecnej te dwa algorytmy stanowią podstawę, na której budowane są nowe algorytmy, wykorzystujące możliwości współczesnych algorytmów (\emph{Czy wymieniać tutaj przykłady algorytmów, które bazują na nich + refy do kilku artykułów?}).

\subsection{Algorytm Apriori}\label{apriori:section}
Pierwszy algorytm odkrywający reguły asocjacyjne został przedstawiony w roku 1994 w pracy~\cite{Apriori:Main}. Niżej przedstawione zostaną szczegóły działania tego algorytmu nazwanego algorytmem \emph{Apriori}.

Zagadnienie odkrywania reguł asocjacyjnych można podzielić na dwa etapy~\cite{Problem:Statement}:
\begin{enumerate}
	\item Odkrywanie zbiorów częstych, których wartość wsparcia jest wyższa od wartości $minsup$.
	\item Generowanie reguł asocjacyjnych na podstawie znalezionych zbiorów częstych. Reguła $X \Rightarrow Y$ jest wynikiem działania algorytmu dla zbioru $Z = X \cup Y$, jeżeli spełnia ona nierówność $conf(X \Rightarrow Y) \geq minconf$. Ponieważ zbiór $Z = X \cup Y$ jest zbiorem częstym, to reguła spełnia również warunek przekraczania minimalnego wsparcia.

	Na tym etapie możliwe jest tworzenie reguł, w których w zbiorze \emph{poprzedników} ($X$ z oznaczeń z definicji~\ref{regula:def}) jest wiele elementów oraz jeden w \emph{następniku} (zbiór $Y$ z definicji~\ref{regula:def})~\cite{Problem:Statement} lub dopuszczana jest możliwość wielu elementów również w następniku~\cite{Apriori:Main}. W niniejszej pracy analizowany jest sposób generowania reguł, w którym oba zbiory mogą być zbiorami wieloelementowymi.
\end{enumerate}

W kolejnych podrozdziałach przedstawione zostaną etapy tworzące razem algorytmy Apriori.

\subsubsection{Generowanie zbiorów częstych}\label{apriori:gen}
W celu wyznaczenia zbiorów częstych algorytm dokonuje analizy bazy danych $DB$, by w kolejnych iteracjach generować rodziny coraz to liczniejszych zbiorów, będących zbiorami częstymi dla zadanej wartości $minsup$. Algorytm zaczyna od znalezienia wszystkich zbiorów jednoelementowych, które są zbiorami częstymi. W każdym kolejnym kroku generowane są zbiory częste na podstawie zbiorów wygenerowanych w kroku poprzednim. Proces ten jest kontynuowany do momentu aż nie zostaną znalezione żadne zbiory częste.

Algorytm generuje zbiory kandydatów jedynie na podstawie zbiorów częstych odkrytych w kroku poprzednim - co ważne generowanie ich odbywa się bez wielokrotnego przeglądania bazy danych transakcji. Intuicja podpowiada, że każdy podzbiór zbioru częstego jest zbiorem częstym. Zatem, każdy zbiór częsty zawierający $k$ elementów może być wygenerowany na podstawie połączenia dwóch zbiorów posiadających $k-1$ elementów, a na koniec kasując te zbiory, których jakikolwiek podzbiór nie jest częsty~\cite{Apriori:Main}.

\begin{table}
	\centering
	\begin{tabular}{|c|p{7.7cm}|} \hline
	$k$-zbiór & Zbiór zawierający $k$ elementów. \\ \hline
	$L_k$ & Zbiór zawierający $k$-zbiory. Każdy zbiór zawarty w $L_k$ zawiera dwa pola: i) zbiór oraz ii) wartość $support$. \\ \hline
	$C_k$ & Zbiór $k$-zbiorów kandydatów (potencjalnych zbiorów częstych). Każdy zbiór zawarty w $C_k$ zawiera dwa pola: i) zbiór oraz ii) wartość $support$. \\ \hline
	\end{tabular}\label{skroty:znaczanie}
	\caption{Oznaczenie skrótów w opisie algorytmu}
\end{table}

Tabela~\ref{skroty:znaczanie} zawiera spis oznaczeń używanych w opisie algorytmu. 

Procedura \proc{Apriori Frequent Set Generaion} przedstawia pseudokod realizujący opisywany w tym rozdziale algorytm generowania zbiorów częstych.

\begin{codebox}
	\Procname{$\proc{Apriori Frequent Set Generaion}$}\label{apriori:listing}
	\li $\id{L_1} \gets \lbrace 1$-zbiory częste $\rbrace$
		\li \For $(k = 2; \id{L_{k-1}} \neq \emptyset; k++)$
		\li \Do
			\li $\id{C_k} \gets aprioriGen(\id{L_{k-1}})$
			\li \textbf{forall} trasakcja $t \in \id{DB}$
			\li \Do
					\li $C_t \gets subset(C_k, t)$
					\li \textbf{forall} kandydat $c \in \id{C_t}$
					\li \Do c.count++
					\End
				\End
			\li $L_k \gets \lbrace c \in C_k | c.count \geq minsup \rbrace$	
		\End
	\li Answer $\gets \bigcup_k L_k $
\end{codebox}

\paragraph{Procedura aprioriGen}
Procedura \id{aprioriGen} reprezentuje proces twórzenia zbiorów $k$-elementowych kandydatów na podstawie zbiorów wejściowych ${k-1}$-elementowych. Procedura ta jest podzielona na dwa etapy: łączenia oraz przycinania. 

Jak łatwo zauważyć wynikiem działania \proc{Join Step} są zbiory $k$-elementowe, które powstały na podstawie zbiorów wejściowych $L_{k-1}$, a ich zawartość różni się tylko jednym elementem - ostatnim. Ważnym faktem jest to, iż elementy w zbiorach są uporządkowane leksykograficzne, co wykorzystywane jest w tej procedurze.

\begin{codebox}
	\Procname{$\proc{Join Step}$}
	\li \textbf{insert into} $C_k$
	\li \textbf{select} p.item$_1$, p.item$_2$, \dots, p.item$_{k-1}$, q.item$_{k-1}$
	\li \textbf{from} $L_{k-1}$ p, $L_{k-1}$ q
	\li \textbf{where} p.item$_1 = $ q.item$_1$, \dots, p.item$_{k-2}$ = q.item$_{k-2}$, p.item$_{k-1}$ $<$ q.item$_{k-1}$
\end{codebox}

Warto zauważyć, że \proc{Join Step} jest ekwiwalentem rozszerania zbioru $L_{k-1}$ każdym elementem zbioru elementów $I$, a następnie kasowania tych $(k-1)$-zbiorów otrzymanych przez usuwanie $(k-1)$ elementu, które nie są w $L_{k-1}$. 

Warunek p.item$_{k-1}$ $<$ q.item$_{k-1}$ zapewnia, że nie będą generowane duplikaty. Dlatego też po etapie łączenia zachodzi zależność $C_k \supseteq L_k$.

Następnym krokiem jest \proc{Prune Step}, w którym usuwane są wszystkie elementy $c \in C_k$, którego jakikolwiek podzbiór $(k-1)$-elementowy zbioru $c$ nie należy do $L_{k-1}$.

\begin{codebox}
	\Procname{$\proc{Prune Step}$}
		\li \textbf{forall} zbiór $c \in C_k$ 
		\li \Do
			\li \textbf{forall} $(k-1)$-podzbiór s zbioru $c$
					\li \Do 
						\li \If $s \notin L_{k-1}$
						\li \Then
							\li \textbf{delete} $c$ z $C_k$
						\End
					\End
		\End
\end{codebox}

Celem operacji przycinania (\english{prune}) jest ograniczenie rozmiaru zbioru $C_k$ przed sprawdzeniem wsparcia dla kandydatów w bazie danych $DB$. W tym celu wykorzystywana jest właściwość, z której wynika, że jeśli jakiś $(k-1)$-podzbiór danego kandydata ($c \in C_k$) nie występuje w $L_{k-1}$, to kandydatk $c$ nie jest zbiorem częstym i powinien być usunięty z $C_k$.

\subsubsection{Generowanie reguł asocjacyjnych}
Po zakończeniu pierwszego etapu algorytm przystępuje do drugiego, czyli do budowania reguł asocjacyjnych na podstawie odkrytych zbiorów. Podobnie, jak w~\cite{Apriori:Main} algorytm będący przedmiotem analizy niniejszej pracy, generuje wszystkie możliwe reguły asocjacyjne dla zadanego zbioru. Mniej ogólny sposób generowania reguł został przedstawiony w pracy~\cite{Problem:Statement}, jednakże podjęto decyzję, że jest to sposób zbyt mało użyteczny w środowisku produkcyjnym.

Aby wygenerować reguły, dla każdego zbioru częstego $l$ znajdowane są niepuste podzbiory - podzbiór taki oznaczony jest jako $a$. Dla takich oznaczeń wygenerowna zostanie reguła $a \Rightarrow (l-a)$, jeżeli spełniona jest nierówność $\frac{support(l)}{support(a)} \geq minconf$. Warto zauważyć, że dla każdego zbioru częstego generowane są wszystkie możliwe niepuste podzbiory - zapewnia to, że odkryte zostaną wszystkie możliwe reguły.

Procedura~\proc{Generate Frequent Itemsets} prezentuje generowanie reguł asocjacyjnych na podstawie odkrytych zbiorów częstych.

\begin{codebox}
	\Procname{$\proc{Generate Frequent Itemsets}$}
		\li \textbf{forall} zbiór częsty $l_k$, $k \geq 2$ \Do
			\li \textbf{call} genrules($l_k$, $l_k$)
					\End
		\End
\end{codebox}


\begin{codebox}
	\Procname{$\proc{Genrules}$($l_k$: $k$-zbiór częsty, $a_m$: $m$-zbiór częsty)}
		\li $\id{A} \gets \lbrace (m-1)$-zbiór $a_{m-1} | a_{m-1} \subset a_m \rbrace$
		\li \For $a_{m-1} \in A$
			\li \Do
			\li $\id{conf} \gets \frac{support(l_k)}{support(a_{m-1})}$
			\li \If $\id{conf} \geq \id{minconf}$
				\li \Then
						\li \textbf{output} reguła $a_{m-1} \Rightarrow (l_k - a_{m-1})$ \\ ufność = $conf$ oraz wsparcie= $support(l_k)$
						\li \If $m-1 > 1$ 
							\li \Then
							\li \textbf{call} genrules($l_k$, $a_{m-1}$) \\ generowanie reguł podzbiorów zbioru $a_{m-1}$
						\End
				\End
			\End
		\End
\end{codebox}

%Kluczową własnością wykorzystywaną w tym etapie jest antymonotoniczność funkcji wsparcia:

%Własność Apriori Niech X i Y będą zbiorami towarów. Jeśli X   Y, to 
%support(X)  support(Y)
%Dowód: X  Y  cover(X)  cover(Y)  support(X)  support(Y).

%Rysunek 3: Ilustracja własności Apriori: jeśli AB nie jest częsty, to możemy wykluczyć wszystkie jego nadzbiory
 
%Antymonotoniczność wsparcia ze względu na zawieranie się zbiorów oznacza, że rozszerzenie nieczęstego zbioru nie może prowadzić do częstego zbioru. Jeżeli zatem k-zbiór X nie jest częsty, wówczas możemy pominąć wszystkie zbiory Y, takie że X Í Y (por. rysunek 3).

%Niżej przedstawiamy szczegóły algorytmu, w którym własność Apriori jest wykorzystywana. Analizujemy, jak rodzina k-zbiorów częstych Lk powstaje z k-1 zbiorów częstych Lk-1.

%Dowolny k-zbiór nazywamy kandydatem, jeśli każdy jego (k-1)-podzbiór jest częsty. 

%Łączenie:
%Zakładamy, że towary w bazie transakcyjnej są ponumerowane i wszystkie zbiory są uporządkowane leksykograficznie w rosnącym porządku. Czyli każdy częsty l Î Lk-1 jest reprezentowany jako tablica 
%l = (l[1], l[2],...,l[k-1])
%gdzie l[1] < l[2] < ... < l[k-1]. Operacja łączenia Lk-1  Lk-1 jest wykonywana przez łączenie wszystkich par (k-1)-zbiorów częstych.

%Dwa zbiory l1,l2 Î Lk-1 zostają połączone, jeśli mają one k-2 takich samych elementów na początku, tzn. 
%(l1[1] = l2[1]) Ù (l1[2] = l2[2]) Ù...Ù(l1[k-2] = l2[k-2]) oraz (l1[k-1] < l2[k-1])
%(1)
%Warunek l1[k-1] < l2[k-1] jest po to, aby zapobiegać powstawaniu powtarzających się kandydatów. Wynikiem łączenia zbiorów l1 i l2 spełniających warunek jest k-elementowy zbiór l, który powstaje przez dołączenie l2[k-1] na końcu l1. 

%Przycinanie: Można zauważyć, że powstający zbiór kandydatów Ck jest nadzbiorem zbioru Lk, tzn. że jego elemnety mogą być częste lub nieczęste, ale wszystkie k-zbiory częste należą do Ck. Celem operacji przycinania jest redukowanie rozmiaru zbioru Ck kandydatów przed sprawdzaniem ich wsparcia w bazie transakcji. W tym celu wykorzystujemy własciwość Apriori, z której wynika, że jeśli jakiś (k-1)-podzbiór danego kandydata nie występuje w Lk-1, to ten kandydat powinien być usunięty z Ck. Algorytm sprawdzania obecności podzbiorów kandydatów w Lk-1 może być efektywnie implementowany za pomocą drzewa haszującego dla częstych zbiorów w Lk-1.

\subsection{Algorytm FP-growth}

Podstawową wadą algorytmu Apriori jest wysoki koszt przetważania dużych zbiorów danych. Przykładowo, dla $10^4$ 1-zbiorów częstych, algorytm Apriori wygeneruje około $10^7$ 2-zbiorów kandydatów, które następnie poddane zostaną weryfikacji, czy są zbiorami częstymi. Fakt ten stał się podstwawowym powodem, dla którego zaprojektowany oraz zaimplementowany został algorytm \emph{FP-growth} w pracy~\cite{Main:FPgrowth}. Algorytm ten pozwala wyeliminować konieczność generowania tak dużej liczby kandydujących zbiorów elementów. 

Algorytm FP-growth można podzielić na trzy podstawowe kroki.
\begin{enumerate}
	\item W kroku pierwszym generowana jest skompresowana wersja bazy danych $DB$, mająca postać drzewa częstych wzorców.W pierwszym dochodzi do kompresji bazy danych, 
	\item Drugim krokiem jest transformacja tak skonstruowanego drzewa do postaci FP-drzewa (patrz definicja~\ref{fptree:def}).
	\item Trzeci krok polega na analizie FP-drzewa celem odnalezienia reguł asocjacyjnych. W kroku tym stosowana jest metoda dziel i zwyciężaj (\english{divide-and-conquer}) zamiast podejścia Apriori. Takie podejście w sposób znaczący zmniejsza rozmiar 
\end{enumerate}

\subsubsection{Definicje}
\begin{df}\label{fptree:def}
FP-drzewo (\english{frequent-pattern tree}) jest to ukorzeniony, etykietowany w wierzchołkach graf acykliczny spełniający poniższe cechy.
\end{df}
\begin{enumerate}
	\item Korzeniem drzewa jest jeden element \id{null}, zbiór poddrzew prefixowanych elementami (jako dzieci elementu \id{null}) oraz tablicy wskaźników \id{element} $\rightarrow$ \emph{wskaźnik na element drzewa}.
	\item Każdy wierzchołek poddrzewa składa się z trzech elementów: nazwy element, licznika (\english{count}) oraz wskaźnika na inny wierzchołek. Nazwa elementu w sposób jednoznaczny identyfikuje element ze zbioru elementów $I$, licznik przechowuje liczbę transakcji reprezentowanych przez ścieżkę od \id{null} do tego elementu, natomiast wskaźnik wskazuje na kolejny wierzchołek w FP-drzewie, którego nazwa jest identyczna do danego.
	\item Każdy wpis w tablicy nagłówkowej (\english{frequent-item-header table}) skałada się z dwóch elementów: nazwy elementu oraz wskaźnika na pierwszy element w drzewie posiadający identyczną nazwę.
\end{enumerate}

\subsubsection{Konstrukcja FP-drzewa}
\textbf{Input:} Baza danych transakcji $DB$ oraz minimalne wsparcie (\id{minsupp}).

\textbf{Output:} FP-drzewo utworzone na podstawie zawartości $DB$

\textbf{Metoda:} Poniżej zostanie opisany proces konstrukcji FP-drzewa.

\begin{enumerate}
	\item Przeskanowanie bazy danych transakcji $DB$ odbywa się jednokrotnie. Utworzony na tej podstawie zostanie zbiór $F$, zawierający 1-zbiory częste. Posortowany malejąco zbiór $F$ na podstawie wartości $support$ dla każdego elementu tworzy listę $FList$, czyli listę wsystkich elementów tworzących jednoelementowe zbiory częste.
	\item f
\end{enumerate}

\subsubsection{Eksploracja FP-drzewa}

%Dzięki wykorzystaniu "downward-closure"~\cite{Problem:Statement} właściwości wsparcia, który gwarantuje, że wszystkie podzbiory zbioru częstego, również są zbiorami częstymi (i na odwrót), możliwe jest w sposób efektywny znalezienie wszystkich zbiorów częstych w zadaniej bazie danych.

%While the second step is straight forward, the first step needs more attention.
%Finding all frequent itemsets in a database is difficult since it involves searching all possible itemsets (item combinations). The set of possible itemsets is the power set over I and has size 2n - 1 (excluding the empty set which is not a valid itemset). Although the size of the powerset grows exponentially in the number of items n in I, efficient search is possible using the downward-closure property of support[2][4] (also called anti-monotonicity "Jian Pei, Jiawei Han, and Laks V.S. Lakshmanan. Mining frequent itemsets with convertible constraints. In Proceedings of the 17th International Conference on Data Engineering, April 2–6, 2001, Heidelberg, Germany, pages 433-442, 2001.") which guarantees that for a frequent itemset, all its subsets are also frequent and thus for an infrequent itemset, all its supersets must also be infrequent. Exploiting this property, efficient algorithms (e.g., Apriori[6] and Eclat[7]) can find all frequent itemsets.

%%%%%%%%%%%%%%%%%%%%%%%%%%% GARBAGE!!!

%Z uwagi na fakt, że należy w pewien sposób ograniczyć liczbę reguł, jako wyniku działania algorytmu, wprowadza się dwie wartości $minsup$ oraz $minconf$. Wszystkie kombinacje przedmiotów ($\mathbf{X}_k$), które spełniają nierówność $sup(\mathbf{X}_k) \leq minsup$ nazywane są zbiorami dużymi (LARGE?). Pozostałe zbiory nazywane są zbiorami małymi (SMALL?)~\cite{Problem:Statement}.

%Ze wzoru wynika, że poziom ufności jest estymacją prawdopodobieństwa tego, że elementy tworzące zbiór $\mathbf{Y}$ zostaną kupione przez danego klienta pod warunkiem, że klient kupi elementy należące do zbioru $\mathbf{X}$.

%Inną motywacją, poza statystyczną ważnością, dlaczego interesujący jest poziom pokrycia, jest fakt, że poszukiwane reguły powinny spełniać pewne wymaganie, co do wysokości wartości $sup$ z powodów biznesowych. Jeśli poziom pokrycia nie jest wystarczająco wysoki, to oznacza to, iż reguła nie jest warta brania pod uwagę, bądź jest ona mniej preferowana (może być rozpatrzona później)~\cite{Problem:Statement}. Dlatego też w przypadku reguły asocjacyjnej określenie wartości $sup$ dla każdej reguły i weryfikowanie tylko tych, które spełniają pewne wymagania co do wysokości tego współczynnika.



%%%%%%%%%%%%%%%%%%%%%%%%%%%
%Reguła asocjacyjna zdefiniowana w Definicji~\ref{regula:def} określa, że jeśli dany klient kupił towary ze zbioru $\mathbf{X}$, najprawdopodobniej kupi też towary ze zbioru $\mathbf{Y}$. Aby reguła asocjacyjna stanowiła interesujące źródło informacji dla analityka stosującego techniki eksploracji danych, musi ona spełniać określone warunki wyrażone za pomocą odpowiednich miar. Dwie najbardziej popularne miary jakości reguł asocjacyjnych to poziom pokrycia (\\english{support}) oraz poziom ufności (\english{confidence}).


%Oznacza to, że poziom pokrycia ($sup$) jest stosunkiem liczby transakcji zawierających elementy sumy zbiorów (czyli $|\mathbf{Z}|$, gdzie $\mathbf{Z} = \mathbf{X} \cup \mathbf{Y}$) do liczby wszystkich transakcji w systemie ($|\mathbf{T}|$). Wzór \ref{support2:def} przedstawia sposó obliczania wartości $sup$.

%\begin{equation}\label{support2:def}
	%sup = \frac{|\mathbf{Z}|}{|\mathbf{T}|}
%\end{equation}
\chapter{Uniwersalna architektura procesorów wielordzeniowych}
Rozdział ten stanowi wprowadzenie to zagadnienia obliczeń wielordzeniowych. Przedstawiona została w nim pokrótce historia rozwoju oprogramowania oraz sprzętu komputerowego, który podążał za co raz to wyższymi wymaganiami stawianymi przez użytkowników. Opisana zostanie również historia rozwoju \emph{procesora} (\acronym{CPU}, \english{Central Processing Unit}) - głównej jednostki obliczeniowej w praktycznie każdym urządzeniu elektronicznym. Później opisany zostanie rozwój karty graficznej oraz zmiana jej zastosowania na przestrzeni lat. 

Ostatnim, lecz nie najmniej ważnym elementem tego rozdziału jest opis uniwersalnej architektury obliczeniowej CUDA (\english{Compute Unified Device Architecture}) wprowadzonej przez firmę nVidia w roku 2007, która umożliwia wykorzystanie mocy obliczeniowej \termdef{procesorów graficznych} (\acronym{GPU}, \english{Graphics Processing Unit}), bądź innych procesorów wielordzeniowych, do rozwiązywania ogólnych problemów obliczeniowych w sposób znacząco wydajniejszy niż w przypadku tradycyjnych, sekwencyjnych procesorów~\cite{cuda:zone}.

Na rysunku~\ref{rys:gpu_cpu} przedstawiony został przyrost w przepustowościach pamięci - odpowiednio dla GPU oraz CPU na przestrzeni lat. Łatwo zauważyć, że w ciągu 7 lat procesor graficzny zyskał około 10-krotną przewagę nad CPU. Dlatego też co raz większe jest zainteresowanie wykorzystaniem kart graficznych w dziedzinach innych niż tylko renderowanie grafiki.

\begin{figure}[h]
\centering\includegraphics[width=0.9\textwidth]{figures/03/gpu_cpu.png}
\caption{Przepustowość pamięci na CPU oraz GPU~\cite{Cuda:PGuide}}\label{rys:gpu_cpu}
\end{figure}

\section{Czasy przetwarzania równoległego}
W poprzednich latach dokonał się znaczący postęp w przechodzeniu przemysłu komputerowego na obliczenia wykonywane równolegle. W roku 2010 większość komputerów konsumenckich była dostarczana do odbiorcy z procesorem zawierającym więcej niż jeden \emph{rdzeń} (\english{core}). Począwszy od procesorów dwurdzeniowych w laptopach do 8- czy 16-rdzeniowych stacji roboczych - od pewnego czasu obliczenia równoległe nie są już tylko domeną superkomputerów lub mainframe'ów (ang. \emph{main} – główny, \emph{frame} – struktura). Co więcej, urządzenia elektroniczne takie jak telefony komórkowe czy też przenośne odtwarzacze muzyki wyposażane są w procesory wielordzeniowe, co zapewnia im możliwości dalece przekraczające te dostępne dla ich poprzedników.

Wynika z tego, że coraz to więcej programistów będzie musiało radzić sobie z implementacją oprogramowania przeznaczonego na jednostki równoległe, wykorzystywać nowe technologie, które będę pozwalały na dostarczenie nowatorskich rozwiązań dla co raz bardziej wymagającej rzeszy użytkowników. Wiersze poleceń to przeżytek - od dawna komputerem steruje się za pomocą skomplikowanych interfejsów graficznych. To samo tyczy się telefonów komórkowych - w chwili obecnej telefon to tylko jedna z wielu funkcji, jakich może dostarczyć współczesny aparat telefoniczny. Teraz telefony mogą jednocześnie grać muzykę, dostarczać informacji o obecnej lokalizacji przy użyciu modułu nawigacji satelitarnej (\acronym{GPS}, \english{Global Positioning System}) i jednocześnie wyświetlać zdjęcia.

\subsection{Procesor}\label{03-procesor}
Przez około 30 lat jedną z ważniejszych metod udoskonalania centralnych jednostek obliczeniowych, a przez to zwiększania komfortu korzystania z urządzenia przez użytkownika, było zwiększanie prędkości z jaką operował zegar procesora. W latach 80 XX wieku procesor przeznaczony na rynek konsumencki operował z prędkością oscylującą w okolicach $1$MHz. Około 30 lat później, w czasach współczesnych, większość komputerów osobistych wyposażona jest w procesory o prędkościach od $1$ do $4$GHz, czyli obecne jednostki są około $1000$ szybsze od wczesnych procesorów. Zwiększanie prędkości, z jaką operuje zegar procesora jest niezawodnym źródłem zwiększania szybkości, należy jednak podkreślić, że nie jest to jedyna metoda na zwiększanie jego wydajności.

Jednakże ograniczenia technologiczne wyznaczają pewne granice, w jakich może wzrastać prędkość zegara procesora. Dlatego też poszukuje się innych, równie niezawodnych źródeł zwiększenia wydajności. Nie można już dłużej polegać jedynie na zwiększaniu prędkości. Z powodu restrykcji na mocy oraz wydzielanym cieple oraz docieraniu do granicy rozmiaru tranzystora, naukowcy oraz producenci rozpoczęli poszukiwanie nowych źródeł i sposobów na zwiększenie możliwości procesorów.

Poza światem konsumentów, czyli w świecie tzw. superkomputerów przez dekady osiągano niezwykle wielkie przyrosty mocy w bardzo podobny sposób. Moc procesora używanego w tych komputerach rosła tak samo szybko, jak w przypadku przyrostów procesorów desktopowych. Jednakże, poza wielkimi przyrostami mocy obliczeniowej na jednej jednostce, producenci superkomputerów tworzyli komputery, w których solidne przyrosty w wydajności osiągano dzięki zwiększaniu liczby używanych procesorów. Nie jest niczym niezwykłym, że pojedynczy superkomputer składa się z dziesiątek lub setek tysięcy procesorów działających równolegle.

W poszukiwaniu dodatkowych możliwości dla komputerów osobistych, poprawa wydajności w przypadku superkomputerów rodzi pytanie: Dlaczego zamiast zwiększać wydajność pojedynczej jednostki, nie umieścić w komputerze osobistym więcej rdzeni? W wypadku zwiększania liczby rdzeni rozwój jednostek obliczeniowych nie byłby ograniczony przez te same niedogodności, co w przypadku ciągłego zwiększania prędkości zegara procesora.

W roku 2005 wiodący producenci prosesorów zaczęli oferować jednostki z dwoma, zamiast z jednym rdzeniem. W latach następnych kontynuowano tę praktykę, tworząc jednostki trzy-, cztero-, sześcio- oraz ośmio-rdzeniowe. Czasami nazywa się ten okres \emph{rewolucją wielordzeniową}~\cite{Cuda:Example}, ponieważ zmiana podejścia do zwiększania wydajności jednostek w znaczący sposób wpłynęła na ewolucję konsumenckiego rynku komputerów.

W chwili obecnej praktycznie każdy komputer osobisty jest wyposażony w procesor dwurdzeniowy. Nawet na rynku niskobudżetowych komputerów z bardzo niskim zapotrzebowaniem na moc, dokonała się rewolucja wielordzeniowa - już nawet netbooki będą wyposażone w dwa rdzenie~\cite{intel:netbook}. 

\section{Wpływ procesorów graficznych na procesy obliczeniowe}

Wydawać by się mogło, że użycie procesora graficznego, jako jednostki obliczeniowej dla problemów nie związanych bezpośrednio z przetwarzaniem grafiki jest podejściem zupełnie nowym. Porównując to do klasycznych obliczeń na procesorach jest to w istocie konepcja nowa. Jednakże obliczenia na jednostkach graficznych nie są tak nowe, jak mogłoby się wydawać na pierwszy rzut oka. 

\subsection{Krótka historia kart graficznych}

W rozdziale~\ref{03-procesor} przedstawiony został rozwój procesora w dwóch płaszczyznach - prędkości oraz liczbie rdzeni. W międzyczasie karty graficzne przeżywały rewolucję rozwojową. Na końcu lat 80 oraz początku 90, wzrost popularności \emph{graficznych interfejsów użytkownika} (\acronym{GUI}, \english{Graphical User Interface}), a w szczególności systemów operacyjnych takich jak Microsoft Windows, wymusił na producentach sprzętu stworzenie nowego typu procesora. We wczesnych latach 90 użytkownicy zaczęli kupować karty graficzne 2D dla swoich komputerów osobistych. Te akceleratory grafiki oferowały wsparcie sprzętowe do operacji bitmapowych by umożliwić wykorzystanie graficznego interfejsu użytkownika.

Mniej więcej w tym samym czasie, w świecie profesjonalnych komputerów, firma o nazwie \emph{Silicon Graphics} popularyzowała w latach 80-tych użycie grafiki trójwymiarowej w wielu dziedzinach, m.in. aplikacje rządowe oraz obronne, aplikacje wspierające naukę oraz wizualizację wyników przeprowadzanych badań naukowych, a także w tworzeniu trójwymiarowych efektów filmowych, czyli rzeczy do tej pory niedostępnych na rynku. W roku 1992 Silicon Graphics stworzyło interfejs programistyczny do swojego sprzętu poprzez wydanie biblioteki IRIS GL, która następnie wyewoluowała do OpenGL (\english{Open Graphics Library}), czyli specyfikacji uniwersalnego API do generowania grafiki. Firma ta wypuszczając tę bibliotekę chciała, by była ona ustandaryzowaną, niezależną od platformy biblioteką do tworzenia aplikacji trójwymiarowych. Zupełnie, jak w pryzpadku licznych rdzeni na procesorach (patrz rozdział~\ref{03-procesor}) było tylko kwestią czasu, aż aplikacje 3D znajdą się w domach użytkowników na ich prywatnych komputerach.

\begin{figure}[h]
\centering\includegraphics[width=0.9\textwidth]{figures/03/doom.png}
\caption{Zrzut ekranu z gry DOOM - klasycznej gry z gatunku FPS}\label{rys:doom}
\end{figure}

Dwa znaczące powody w połowie lat 90-tych spowodowały, że pojawiła się nagła potrzeba stworzenia procesora graficznego potrafiącego tworzyć trójwymiarowe efekty (\acronym{3D}, \english{three-dimensional}). Pierwszym było pojawienie się wciągających gier, w których gracz ogląda świat gry z perspektywy pierwszej osoby, czyli oczami bohatera (\acronym{FPP}, \english{First Person Perspective}), takich jak Doom (patrz rysunek~\ref{rys:doom}), Duke Nukem 3D oraz Quake. Przyczyniło się to do rozpoczęcia tworzenia bardziej realistycznych środowisk trójwymiarowych, w których możliwe byłoby tworzenie gier komputerowych. Pomimo tego, że grafika 3D mogła być wykorzystana w praktycznie wszystkich grach komputerowych, to wyjątkowa popularność gier z gatunku FPS (\english{First Person Shooter}) spowodowała, że większy nacisk położono na prywatne jednostki komputerowe. W tym samym czasie firmy takie jak NVIDIA, ATI Technologies oraz 3dfx Interactive zaczęły produkować akceleratory graficzne, które były w stanie sprostać ciągle rosnącym wymaganiom klientów. Te strategiczne decyzje spowodowały, że grafika trójwymiarowa na dobre zadomowiła się na komputerach stacjonarnych i w nadchodzących latach to ona miała wieść prym wśród technologii komputerowych.

\begin{figure}[h]
\centering\includegraphics[width=0.7\textwidth]{figures/03/geforce-256.jpg}
\caption{Karta graficzna NVIDIA GeForce 256}\label{rys:gf256}
\end{figure}

Wypuszczenie w drugiej połowie roku 1999 na rynek karty graficznej NVIDIA GeForce 256 przesunęło możliwości komputerów osobistych jeszcze dalej~\cite{nvidia:geforce256} - wygląd tej karty przedstawiony został na rysunku~\ref{rys:gf256}. Po raz pierwszy obliczenia świetlne oraz transformacje obliczeniowe możliwe były do wykonania bezpośrednio na procesorze graficznym, co dawało jeszcze większe możliwości tworzenia aplikacji atrakcyjnych wizualnie. Ponieważ operacje te były już integralną częścią \emph{potoku wywołań} (\english{pipeline}) biblioteki OpenGL, karta GeForce 256 wyznaczyła początek progresywnego zwiększania operacji dostępnych w biliotece OpenGL, które były implementowane na procesorze graficznym.

Uważa się, że wypuszczona w roku 2001 seria kart graficznych GeForce 3 jest jednym z większych przełomów w świecie technologii GPU~\cite{nvidia:geforce3,Cuda:Example}. Seria ta była pierwszą w świecie, której chip był kompatybilny z nowym w ówczesnym czasie standardem firmy Microsoft o nazwie DirextX 8.0, który wymagał by hardware zawierało \emph{programowalne wierzchołki} (\english{programmable vertex}) oraz \emph{programowalne cieniowanie} (\english{programmable pixel shading}) w kolejnych fazach przetwarzania. Po raz pierwszy programiści mogli mieć jakąkolwiek kontrolę nad obliczeniami dokonywanymi bezpośrednio na karcie graficznej.

\subsection{Wczesne obliczenia na kartach graficznych}
\label{chapter:wczesne}

Wyprodukowanie kart, które posiadały następujące po sobie programowalne fazy (potok wywołań) spowodowały, że wielu naukowców zaczęło wykorzystywać ich możliwości nie tylko poprzez używanie OpenGL czy też DirectX do standardowych zadań. Takie podejście do obliczeń na kartach graficznych we wczesnych latach obliczeń na GPU było niezwykle skomplikowane. Ponieważ standardowe API graficzne takie jak OpenGL, czy też DirectX były jedynymi metodami do interakcji z GPU, to programowanie obliczeń na kartach graficznych nadal sprowadzało się do implementowania przetwarzania graficznego poprzez dotępne metody interfejsu programistycznego karty. Z tego też powodu, wielu naukowców wykonywało swoje obliczenia poprzez wspomniane wcześniej API w taki sposób, by ich problemy sprowadzone zostały do renderowania grafiki, a następnie odpowiedniego przetworzenia.

Zasadniczo każda karta graficzna we wczesnym roku 2000 zaprojektowana była w taki sposób, by produkować kolor dla każdego piksela na ekranie używając programowalnych jednostek arytmetycznych nazywanych \emph{pixel shader}. W ogólności jednostka ta wykorzystuje pozycję (jako parę $(x, y)$) piksela oraz pewne dodatkowe informacje by obliczyć kolor danego piksela na ekranie. Tymi dodatkowymi informacjami mogą być kolory, wymiary tekstur oraz parametry, które mogą zostać podane, bądź obliczone w trakcie interakcji z użytkownikiem, bądź otoczeniem piksela. Ponieważ obliczenia wykonywane na wejściowych kolorach oraz teksturach były kontrolowane przez programistę, zauważono, że te wejściowe ,,kolory'' mogą w rzeczywistości reprezentować dowolną daną.

Zatem jeśli wejście programu było dowolną daną reprezentowaną jako wartość reprezentującą coś zupełnie innego niż wartość, to programiści mieli możliwość wykorzystywania pixel shaderów, czyli krótkiego programu komputerowego w specjalnym języku, do wykonania na tych danych porządanych obliczeń oraz przekształceń. Rezultatem tych obliczeń był ,,kolor'', który w danym kontekscie oznaczał coś zupełnie innego niż tylko kolor na ekranie w dosłownym tego słowa znaczeniu. Można więc nazwać to ,,oszukiwaniem'' karty graficznej poprzez poddawanie danych wejściowych przetwarzaniu graficznemu, jakby były to zwyczajne dane potrzebne do wyrenderowania obrazu. Takie podejście odznaczało się wyjątkowym poziomem pomysłowości, ale niestety wykonanie takich obliczeń było wyjątkowo zagmatwane i skomplikowane.

Z powodu wysokiej przepustowości obliczeń arytmetycznych na GPU, początkowe rezultaty takich eksperymentów obiecywały świetlaną przyszłość obliczeń na jednostkach graficznych. Jednakże model programistyczny stosowany do implementacji takich obliczeń był zbyt restrykcyjny dla rzeszy developerów by mógł być wykorzystywany na szeroką skalę. Możliwe było jedynie wykorzysytwanie ,,kolorów'' oraz podmian tekstur, co w wielu przypadkach stanowiło duże ograniczenie przy bardziej skomplikowanych obliczeniach. Było również sporo ograniczeń do do tego jak i gdzie programista mógł zapisywać wyniki do pamięci karty, tak więc algorytmy wykorzystujące rozproszone lokacje nie były możliwe do implementacji na GPU. Co więcej, bardzo trudno było implementować algorytmy korzystające z obliczeń zmiennopozycyjnych (\english{floating-point}), ponieważ nie można było przewidzieć, jak karta graficzna, jeśli w ogóle, wykona te obliczenia~\cite{Cuda:Example}. Ostatnim z zasadniczych ograniczeń był fakt braku jakiejkolwiek metody do debugowania kodu uruchamianego na karcie graficznej - na przykład w momencie wyraźnie błędnych obliczeń, zawieszania się komputera czy też w momencie nie zakończenia się programu.

Jeśli wcześniej wymienione ograniczenia nie były przeszkodą, ktokolwieek kto chciał wykorzystać moc obliczeniową karty raficznej do wykonywania obliczeń nie związanych z przetwarzaniem grafiki musiał w dalszym ciągu nauczyć się OpenGL bądź DirectX, gdyż to one pozostawały jedynymi metodami do interakcji z GPU. Nie tylko oznaczało to przechowywanie wyników w teksturach graficznych oraz wykonywania obliczeń poprzez wywoływanie funkcji OpenGL lub DirectX, ale dodatkowo oznaczało to pisanie obliczeń w specjalnych językach programowania graficznego (\english{shading languages}). Wymaganie by naukowcy dokonywali obliczeń na ograniczonych zasobach oraz specjalnych językach programowania oraz nauczenia się specyfiki jednostek graficznych przed przystąpieniem do wykorzystania możliwości GPU do obliczeń spowodowało, że metoda ta nie przyjęła się, jako wiodąca w świecie naukowców.

\section{CUDA}

Dopiero po około pięciu latach od wydania serii 3 GeForce obliczenia na jednostkach graficznych były gotowe rzeczywiście być implementowane przez programistów z pełnym wykorzystaniem możliwośći kart. W listopadzie 2006 roku NVIDIA wypuściła na rynek pierwszą kartę graficzną wspierającą DirectX 10 - GeForce 8800 GTX~\cite{nvidia:geforce8800}. Karta ta była również pierwszą jednostką zbudowaną na architekturze CUDA. Architektura ta zawierałą nowe komponenty zaprojektowane specjalnie pod obliczenia wykonywane na GPU oraz eliminowała wiele ograniczeń, które powstrzymywały programistów przed wykorzystywaniem możliwości karty do obliczeń nie związanych z przetwarzaniem grafiki.

\subsection{Czym jest architektura CUDA?}

W przeciwieństwie do poprzednich generacji kart graficznych, w których jednostki były podzielone na pixel i vertex shader'y, w kartach zgodnych z architektura CUDA zawarta jest zunifikowana linia shaderów (\english{shader pipeline}). Pozwala to każdej jednostce arytmetyczno-logicznej (\acronym{ALU}, \english{Arithmetic Logic Unit} lub \emph{Arithmetic and Logical Unit}) na czipie być wykorzystaną przez program wykonujący podstawowe operacje obliczeniowe.

Ponieważ firma NVIDIA zamierzała tę nową rodzinę kart graficznych była wykorzystywana w innych rejonach niż tylko wyświetlanie grafiki, wspomniane wcześniej jednostki ALU są zbudowane w oparciu o wymagania IEEE (\english{Institute of Electrical and Electronics Engineers}) co do arytmetyki zmiennopozycyjnej. Zostały one również tak zaprojektowane, by zbiór dostępnych operacji był dostosowany do ogólnego użycia niż tylko do wyświetlania grafiki. 

Co więcej, jednostki egzekucyjne na GPU zostały wyposażone w możliwość czytania oraz zapisu do pamięci, a także do \emph{pamięci podręcznej} (\english{cache}) programu, znanej również jako \emph{pamięć współdzielona} (\english{shared memory}).

Wszystkie te usprawnienia wprowadzone w architekturze CUDA celem rozszerzenia możliwości kart graficznych poza jedynie wyświetlanie grafiki~\cite{Cuda:Example}.

\subsection{Od wyświetlania grafiki do obliczeń ogólnego przeznaczenia}
Jak przedstawiono na rysunkach~\ref{rys:gpu_cpu} oraz~\ref{rys:gpu_cpu2} procesory graficzne rozwijały się w kierunku przetwarzania równoległego - głównie ze względu wymagań stawianych przez konsumentów na realistyczne grafiki 3D realizowane w czasie rzeczywistym. Z tego też powodu GPU stały się wyspecjalizowanymi jednostkami o wysokim stopniu zrównoleglenia zadań, wielordzeniowymi oraz przetwarzającymi informacje w wielu wątkach. 

\begin{figure}[h]
\centering\includegraphics[width=0.9\textwidth]{figures/03/gpu_cpu2.png}
\caption{Operacje zmiennopozycyjne na sekundę na GPU oraz CPU~\cite{Cuda:PGuide}}\label{rys:gpu_cpu2}
\end{figure}

Powodem tak dużej rozbieżności w możliwościach przetwarzania równoległego (zauważalne na rysunku~\ref{rys:gpu_cpu2}) pomiędzy CPU a GPU jest fakt wysokiej specjalizacji na intensywne obliczenia w sposób równoległy (czyli to, co jest głównym składnikiem renderowania grafiki) - dlatego też w jednostkach kart grafiki więcej tranzystorów przeznaczonych jest do przetwarzania danych zamiast na cache danych oraz przepływ sterowania - schematycznie zilustrowane to zostało na rysunku~\ref{rys:gpu_cpu_trans}.

\begin{figure}[h]
\centering\includegraphics[width=1.0\textwidth]{figures/03/gpu_cpu_trans.png}
\caption{Schematyczne przedstawienie budowy CPU oraz GPU~\cite{Cuda:PGuide}}\label{rys:gpu_cpu_trans}
\end{figure}

Opisując bardziej szczegółowo, to GPU jest tak zaprojektowane by spełniać wymagania problemów, które mogą być zapisane jako obliczenia równoległę - ten sam program (metoda) jest wykonywany na każdym elemencie danych rónolegle. Ponieważ ten sam program jest uruchomiony na każdym elemencie danyh, to istnieje mniejsze wymaganie w kontroli skomplikowanego przepływu danych. Dodatkowo z powodu wysokiej intensywności arytmetycznej, to opóźnienie dostępu do pamięci może być ukryte przy pomocy obliczeń zamiast dużych części danych przechowywanych w cache jednostki.

Równoległe przetwarzanie danych mapuje elementy danych na równoległe wątki przetwarzające te dane. Wiele aplikacji, które przetwarzają duże zbiory danych może używać podejścia równoległego przetwarzania danych do przyspieszenia obliczeń. W renderowaniu grafiki 3D duże zbiory pikseli oraz wektory są mapowane na równoległe wątki. Podobnie w aplikacjach przetwarzających obraz oraz media, takie jak post-przetwarzanie wyrenderowanych obraów, kodowanie i  dekodowaniu video, skalowanie obrazó, czy też rozpoznawanie wzorców może mapować bloki obrazu oraz pikseli na równoległe wątki. W rzeczywistości, wiele algorytmów poza dziedziną przetwarzania obrazów są przyspieszane poprzez równoległe przetwarzanie danych - począwszy od przetwarzania sygnałów czy też symulacji fizycznych do obliczeń finansowych, czy też biologii oblieczniowej.

\subsection{Skalowalny model programistyczny}
Nadejście wielokorowych procesorów oraz multikorowych GPU oznacza, że główne chipy procesorowe są teraz systemami równoległymi. Co więcej równoległość kontynuuje skalowalność zgodnie z prawem Moora - czyli zgodnie z empirycznym prawem, które polega na obserwacji, że ekonomicznie optymalna liczba tranzystorów w układzie scalonym w kolejnych latach posiada trend wykładniczy (podwaja się w niemal równych odcinkach czasu). Wyzwaniem jest stworzenie takiego oprogramowanie, które w niezuważalny sposó będzie skalowało wykorzystywanie równoległości do zmieniającego się środowiska, w którym wzrastać będzie liczba korów procesora, zupełnie jak aplikacje graficzne, które transparentnie skalują swój paralelizm do wielu korów GPU, niezależnie od tego na jakiej karcie graficznej są one uruchamiane, co wiąże sie bezpośrednio z liczbą dostępnych jednostek na procesorze.

Równoległy model programistyczny CUDA jest stworzony by przezwyciężyć wyzwanie, o którym wspomniano wyżej, z jednoczesnym ograniczeniem czasu potrzebnego programistom na nauczenie się nowego podejścia poprzez wykorzystywanie do tworzenia aplikacji standardowych języków programowania, takich jak język C.

W swoim źródle CUDA posiada trzy kluczowe elementy: hierarchię grup wątków, współdzieloną pamięć oraz barierę synchronizacyjną. Elementy te są udostępnione programiście poprzez ograniczony zbiór rozszerzeń języka programowania.

Te abstrakcyjne elementy dostarczają drobnoziarnistą równoległość na poziomie danych oraz wątków, zagnieżdżone w gruboziarnistej równoległości danch oraz zadań (\english{task}). Takie podejście prowadzi programistę do podziału problemu na większe podproblemy, które mogą być rozwiązane niezależnie przez bloki wątków, a każdy podproblem może być podzielony na mniejsze, które mogą być rozwiązywane wspólnie przez wątki wewnątrz jednego bloku.

Taka dekompozycja zachowuje wszystkie dostępne funkcjonalności języka programowania umożliwiając wątkom kooperować w trakcie rozwiązywania każdego z podproblemów, a w tym samym czasie udostępnia automatyczną skalowalność programu. W rzeczywistości, każdy blok wątków może być zaplanowany na każdym z dostępnych korów na procesorze, w dowolnym porządku, jednocześnie lub sekwencyjnie, tak że skompilowany program w CUDA może być wykonany na dowolnej liczbnie korów procesora, jak zostało to zilustrowane na rysunku~\ref{rys:threads_gpu}. Jedynie \emph{system uruchomieniowy} (\english{runtime system}) musi znać faktyczną liczbę jednostek procesora. Wielowątkowy program jest podzielony na bloki wątkó, które wykonywane są niezależnie jeden od drugiego, dlatego też GPU z większą liczbą korów będzie automatycznie wykonywać program w krótszym czasie niż karta graficzna z mniejszą liczbą jednostek.

\begin{figure}[h]
\centering\includegraphics[width=1.0\textwidth]{figures/03/threads_gpu.png}
\caption{Zilustrowanie automatycznej skalowalności~\cite{Cuda:PGuide}}\label{rys:threads_gpu}
\end{figure}

Ten skalowalny model programistyczny umożliwia architekturze CUDA być stosowaną w szerokim spektrum zastosowań: od wysokowydajnych kart graficznych GeForce czy też na profesjonalnych kartach Quadro na komputerach entuzjastów, kartach Tesla przeznaczonych tylko do obliczeń, do kart GeForce przeznaczonych na domowe komputery.

\subsection{Używanie architektury CUDA}

Wysiłek wprowadzenia nowej architektury przez firmę NIDIA, która łączy możliwości klasycznego sposobu genercji grafiki z nowym podejściem do uniwersalnch obliczeń nie oznacza, że wprowadzono jedynie nową architekturę, bez odpowiedniego podejścia dla programowania. Bez względu na to, ile nowych funkcjonalności NVIDIA doda do swoich chipów, niemożliwe stało się używanie ich w dotychczasowy sposób - tj. poprzez OpenGL lub DirectX. Nie tylko wymagałoby to od użytkowników dalszego przekształcania swoich algorytmów do problemów renderowania grafiki (problem opisany w części~\ref{chapter:wczesne}), ale również w dalszym ciągu użytkownicy (programiści) musieli by programować swoje obliczenia w zorientowanych na renderowanie grafiki środowiskach (tzw. \emph{shading languages}) takich jak GLSL z OpenGL'a czy też HLSL firmy Microsoft.

By osiągnąć możliwie wysoki poziom developerów korzystających z możliwości nowej architektury, NVIDIA rozszeszyła jeden z najbardziej znanych języków programownia - język C, dodając do niego relatywnie mały zbiór nowych słów kluczowych, by programiści mogli korzystać z nowych, specjalnych funkcji dostępnych jedynie na architekturze CUDA. Kilka miesiecy po wprowadzeniu karty GeForce 8800 GTX~\cite{nvidia:geforce8800}, NVIDIA wprowadziła na rynek darmowy kompilator dla języka CUDA C~\cite{cuda:downloads}. W tym samym momencie, CUDA C stał się pierwszym językiem zaprojektowanym przez firmę produkującą sprzęt graficzny w celu wypełnienia luki w obliczeniach ogólnego przeznaczenia realizowanych na kartach graficznych.

Oprócz języka, w którym tworzone mogą być programy korzystające z mocy GPU, NVIDIA dostarczyła również specjalne \emph{sterowniki} (\english{drivers}) by wykorzystać pełną moc architektury CUDA. Użytkownicy nie są już więcej zobowiązani do znania interfejsów programowania specyficznych dla procesorów graficznych (takich jak OpenGL czy też DirectX), a co więcej nie są już oni zmuszeni do sprowadzanie swoich problemów do problemu renderowania grafiki.
\chapter{Technologie}\label{chap:technologie}

Niniejszy rozdział zawiera przegląd oraz krótki opis technologii użytych w~procesie realizacji projektu. Przedstawiono najważniejsze cechy każdej z~nich, pomijając szczegóły, wprowadzając jedynie zarys informacji na temat poszczególnych komponentów, na których budowana była aplikacja. Dokonano również podstawowego wprowadzenia do każdej z~wymienionych technologii, co jest potrzebne do pełnego przedstawienia sposobu ich wykorzystania.

\section{Język C\# i platforma .NET}\label{sec:cs}
Język C\# oraz .NET zostały pierwszy raz przedstawione podczas międzynarodowej konferencji dla programistów o nazwie \emph{Professional Developers Conderence} (\acronym{PDC}) w lipcu roku 2000~\cite{ms:initDotNet} - choć środowisko programistów słyszło o planach firmy Microsoft znacznie wcześniej, jednakże pod różnymi nazwami (m.in. COOL, COM3, Lightning). Pomimo sporego w owym czasie szumu wokół niedoskonałych systemów operacyjnych (np. Windows Me), czy też sporych opóźnień w dostarczaniu owych rozwiązań, firma zdecydowała się na stworzenie własnej platformy programistycznej~\cite{cSharp:inDepthF}.

Należy również podkreślić, że firma Microsoft nie była jedyną, która wspierała tworzenie języka - oprócz niej w sponsoring tego języka włączył się firmy Helwett-Packard oraz Intel. Język C\# jest, podobnie, jak język Java, otwartym językiem, jednakże w zupełnie innym tego słowa znaczeniu. Microsoft sprzyja ścieżce standaryzacyjnej języka, gdy Sun (firma posiadająca prawa do platformy Java) stopniowo udostępnia kod źródłowy Javy na zasadach \emph{wolnego oprogramowania} (\english{open source}) i dopuszczając, a nawet zachęcając do tworzenia innych \emph{środowisk uruchomieniowych} (\english{runtime environments}) Javy. Istnieją alternatywne implementacje CLI oraz C\# (najbardziej znanym jest Mono~\cite{cs:mono}), jednakże należy podkreślić, że nie obejmują one swoją implementacją wszystkiego, co kryje się pod nazwą Microsoft .NET Framework. Ostatecznie język C\# oraz .NET zostały udostępnione w roku 2002, razem z narzędziem programistycznym Vistual Studio .NET 2002.

\subsection{Platforma .NET}
Kiedy wprowadzono termin ''.NET'', został on wcielony w szereg różnych technologii wychodzących spod sztandarów firmy Micorosft. Na przykład Windows Live ID był nazywany .NET Passport, pomimo tego, że nie istniało żadne bezpoźrednie połączenie pomiędzy tą technologią, a tym co jest obecnie znane, jako .NET. Na całe szczeście z czasem zaprzestano używania nazwy .NET w innych usługach i produktach firmy z Redmond. W chwili obecnej nazwa ''.NET'' wiąże się jedynie ze światem programowania.

Należy podkreślić, że .NET zawiera wspólną specyfikację języka (\acronym{CLS}, \english{Common Language Specification}), która udostępnia zestaw podstawowych reguł, niezbędnch do integracji języków na jednej platformie. CLS określa minimalne wymagania, jakie musi spełniać język z rodziny .NET. Kompilatory zgodne z CLS tworzą obiekty, które mogą ze sobą współpracować.

Poniżej przedstawione zostaną trzy składniki, z jakich zbudowana jest platforma .NET oraz łączące je zależności.

\subsubsection{Język, środowisko uruchomieniowe i biblioteki}
W skład platformy .NET wchodzą trzy podstawowe składniki: język programowania, biblioteki oraz środowisko uruchomieniowe. Wprawdzie rozróżnienie tych trzech nie zawsze jest możliwe, to ważna jest świadomość istnienia wyraźnych różnic pomiędzy tymi trzema składnikami.

\paragraph{Język}
Język C\# jest zdefiniowany przez swoją specyfikację (patrz~\cite{cSharp:spec}), która opisuje format kodu programu w C\#, obejmując jednocześnie składnię oraz zachowanie. Nie jest tam zawarta natomiast informacja o platformie, na jakiej zostanie on uruchomiony, poza kilkoma kilkoma aspektami współpracy kompilatora z platformą. 

W teorii dowolna platforma, która wspiera wymagane funkcje (opisane w specyfikacji), może posiadać kompilator, który będzie budował oprogramowanie przeznaczone dla danej platformy.  Na przykład, kompilator języka C\# mógłby produkować na podstawie kodu źródłowego dowolny inny format niż tzw. \emph{język pośredni} (\acronym{IL}, \english{Intermediate Language}).

\paragraph{Środowisko uruchomieniowe}
Środowisko uruchomieniowe jest częścią .NET odpowiedzialną za dokonanie wszystkiego, by uruchomiony kod IL działał zgodnie ze specyfikacją jęzka. W implementacji firmy Microsoft srodowisko uruchomieniowe nazywa się \emph{Common Language Runtime}, czyli w skrócie \acronym{CLR}.

\paragraph{Biblioteki}
\emph{Biblioteki} (\english{libraries}) dostarczają kod, który może być wykorzystywany przez oprogramowanie. Większość bibliotek frameworka .NET są gotowymi produktami języka IL (patrz punkt~\ref{sec:msil}) z \emph{kodem natywnym} (\english{native code}) wykorzystywanym tylko wtedy, gdy jest to potrzbne. Należy podkreślić, że kodu bilbiotek jest znacznie więcej niż kodu środowiska uruchomieniowego. W taki sam sposób można spojrzeć na samochód (biblioteki), który jest znacznie bardziej skomplikowaną jednostką niż sam jego silnik (środowisko uruchomieniowe).

Warto podkreślić, że istnieje podział na biblioteki standardowe oraz pozostałe, które nie zostały ujęte w specyfikacji języka~\cite{cSharp:spec}. Pisząc program, który wykorzystuje tylko te pierwsze można mieć dużą pewność, że będzie on możliwy do uruchomienia na dowolnej implementacji platrofmy - czy to Mono, .NET, czy jakiejkolwiek innej.

Termin \emph{.NET} odnosi się do kombinacji środowiska uruchomieniowego wraz z bibliotekami dostarczanymi przez Microsoft, a także kompilatory C\# oraz VB.NET. .NET moze być pojmowany, jako cała \emph{platforma programistyczna} (\english{development platform}) zbudowana nad systemem Windows.


\subsection{Kompilacja i język MSIL}\label{sec:msil}
Programy pisane na platformie .NET nie są kompilowane do plików wykonywalnych, jak to ma miejsce w przypadku chociażby języka C++. Są one kompilowane do podzespołów, które składają się z instrukcji standardowego \emph{języka pośredniego Microsoft} (\acronym{MSIL}, \english{Microsoft Intermediate Language}). Środowisko CLR dokonuje przekształcenia instrukcji tego języka na kod maszynowy i wykonuje je. Pliki MSIL (często zamiennie używa się skrótowej nazwy IL) stworzone przez kompilator języka C\# są niemalże takie same, jak pliki IL innych języków z rodziny .NET. Kluczową cechą środowiska CLR jest to, że jest ono wspólne, co oznacza, że programy napisne w języku VB.NET czy też C\# są obsługiwane przez te samo środowisko uruchomieniowe.

Kompilacja kodu w języku C\# do IL odbywa się w momencie budowania projektu. W wyniku czego kod IL jest zapisywany na dysku w postaci plików binarnych. W momencie uruchomienia programu zachodzi ponowna kompilacja - tym razem kodu IL. Jest to tak zwana kompilacja \acronym{JIT} (\english{Just In Time}), w wyniku której powstaje kod maszynowy, wykonywany przez procesor maszyny, na której program został uruchomiony.

Kompilatory JIT, standardowo, są uruchamiana \emph{na żądanie} (\english{on demand}) - tzn. w momencie wywołania metody, kompilator JIT analizuje kod IL i tworzy bardzo wydajny (dzięki licznym optymalizacjom dokonywanym na tym etapie) kod maszynowy. Gdy aplikacja jest uruchomiona, kompilacja zachodzi jedynie wtedy, kiedy jest to potrzebne, a po kompilacji JIT kod znajduje się w pamięci podręcznej, przez co możliwe jest jego szybkie wykorzystywanie w późniejszych etapach przetwarzania. Wynika z tego, że aplikacja działa tym szybciej, im dłużej jest uruchomiona, ponieważ możliwe jest wykorzystanie większej ilości skompilowanego kodu.

Specyfikacja CLS wymusza podobieństwo kodu IL w przypadku każdego języka z rodziny .NET. Wynika z tego bezpośrednio, że obiekty utworzone w jednym języku mogą być wykorzystywane w programach napisanych w innych językach (z rodziny .NET oczywisćie). Oznacza to np. fakt, że możliwe jest przygotowanie klasy bazowej w języku VB.NET oraz utworzenie od niej klas pochodnych w programie napisanej w języku C\#.

\subsection{Język programowania C\#}
Język C\# jest niezwykle wydajnym narzędziem do implementacji współczesnych technik programistyznych. Jest on językiem obsługującym strukturalne, oparte na komponentach i obiektowe programowanie, czego należy oczekiwać od współczesnego języka, zbudowanego na doświadczeniach z C++ oraz Java. 

Programiści mówiący o nowej wersji .NET, mają zazwyczaj na myśli ważne wydania fremework'a. W większości przypadków wydanie nowej wersji platrofmy jest połączone z wydaniem Visual Studio, czyli zintegrowanego środowiska programistycznego firmy Microsoft. W tabeli~\ref{dotNet:dates} zebrane zostały informacje o poszczególnych wersjach, które były wydawane wraz z jakimi komponentami oraz kiedy.

\begin{table}
	\centering
	\begin{tabular}{|l|l|l|l|l|} \hline
	\textbf{Data} & \textbf{Framework} & \textbf{Visual Studio} & \textbf{C\#} & \textbf{CLR} \\ \hline
	Luty 2002 & $1.0$ & 2002 & $1.0$ & $1.0$ \\ 
	Kwiecień 2003 & $1.1$ & 2003 & $1.2$ & $1.1$ \\ 
	Listopad 2005 & $2.0$ & 2005 & $2.0$ & $2.0$ \\ 
	Listopad 2006 & $3.0$ & (rozszerzenia do wersji 2005) & [brak] & $2.0$ \\ 
	Listopad 2007 & $3.5$ & 2008 & $3.0$ & $2.0$ SP1 \\ 
	Kwiecień 2010 & $4.0$ & 2010 & $4.0$ & $4.0$ \\ \hline
	\end{tabular}
	\caption{Poszczególne wersje .NET Framework oraz ich składniki~\cite{cSharp:inDepthS}\label{dotNet:dates}}
\end{table}

\paragraph{Obiektowość}
\definicja{Programowanie obiektowe} (\akronim{OOP}, \english{Object Oriented Programming}) jest to paradygmat programowania, w~którym program definiuje się przy pomocy obiektów.
Podstawowe cechy języka, który realizuje paradygmat programowania obiektowego przedstawił Alan Kay w~odniesieniu do języka \definicja{Smalltalk}, pierwszego poprawnie zrealizowanego języka obiektowego, a~tym samym jednego z~poprzedników C\#. Cechy te opisują czyste podejście obiektowe.
\begin{description}
	\item \emph{Wszystko jest obiektem.} Obiekt można przedstawić jako specjalną zmienną, która nie tylko zawiera dane, ale może również realizować żądania, czyli wykonywać na swoich danych pewne ściśle określone operacje. Teoretycznie każdy element świata rzeczywistego, np. samochód, dom, zamek, pracownik, może być reprezentowany w~programie przy pomocy tak skonstruowanego obiektu.
	\item \emph{Aplikacja jest zbiorem komunikujących się między sobą obiektów.} Przesyłanie komunikatów do obiektu to żądanie od niego wykonania pewnej operacji. Można to nazwać wykonaniem funkcji należącej do konkretnego obiektu. Przykładem może być następujący scenariusz: obiekt \emph{Kierowca} wysyła do obiektu \emph{Samochód} komunikat \emph{przyspiesz}, co powoduję wykonanie operacji zwiększenia prędkości.
	\item \emph{Każdy obiekt posiada własną pamięć, na którą składają się inne obiekty.} Tworzenie nowego obiektu polega na łączeniu w~jeden element grupy już istniejących obiektów. Powstaje w~ten sposób wielowarstwowa aplikacja, która jednocześnie ukrywa swoją złożoność za prostymi obiektami. Przykładowo obiekt \emph{Samochód} można zbudować z~następujących obiektów: \emph{Karoseria}, \emph{Koło}, \emph{Silnik} itd. Jednocześnie chcąc zwiększyć prędkość pojazdu odwołujemy się do obiektu \emph{Samochód}, a~nie bezpośrednio do obiektu \emph{Silnik}, który odpowiada za jego prędkość. Dopiero akcja \emph{przyspiesz} wykonana przez obiekt Samochód wykonuje odpowiednie operacje na poszczególnych podzespołach, z których się składa.
	\item \emph{Każdy obiekt posiada swój typ.} W~odniesieniu do obiektu słowo \emph{typ} można zastąpić słowem \definicja{klasa}. Każdy obiekt jest instancją pewnej klasy, której głównym zadaniem jest zdefiniowanie jakie komunikaty można wysłać do obiektu będącego jej egzemplarzem.
	\item \emph{Wszystkie obiekty danego typu obsługują te same komunikaty.} Każdy obiekt danej klasy obsługuje wszystkie komunikaty zdefiniowane w~klasie. Dodatkowo jeśli np. obiekt typu \emph{Student} jest jednocześnie obiektem typu \emph{Człowiek}, to obsługuje wszystkie komunikaty zdefiniowane dla typu \emph{Człowiek}. Umożliwia to pisanie bardziej uniwersalnego kodu, który będzie obsługiwał wszystkie obiekty pasujące do typu \emph{Człowiek}~\cite{cSharp:progr}.
\end{description}

Współcześnie, aby język programowania został uznany za obiektowy musi charakteryzować się wymienionymi poniżej cechami.
\begin{description}
	\item[Abstrakcja] Każdy obiekt systemu jest widziany jako abstrakcyjny ,,wykonawca'', który może realizować pracę, określać i~zmieniać swój stan oraz wysyłać komunikaty do innych obiektów w~systemie. Jednocześnie nie ujawnia on jak zostały zaimplementowane dane cechy.
	\item[Hermetyzacja] Czyli ukrywanie wewnętrznej implementacji obiektu. Gwarantuje to, że obiekt nie ma możliwości zmiany stanu wewnętrznego innego obiektu w~nieprzewidziany przez programistę sposób. Tylko metody (funkcje) składowe obiektu mają prawo do modyfikacji jego stanu. Dodatkowo pozwala to na swobodną modyfikację kodu przez programistę o~ile nie zmieniają się metody składowe. Przykładowo w~pierwszej wersji klasy programista może zaimplementować listę jako tablicę o~stałym rozmiarze, natomiast w~kolejnej wersji zmienić tablicę na listę jednokierunkową.
	\item[Polimorfizm] Referencje mogą wskazywać na obiekty różnego, ale zgodnego typu. Wywołanie metody dla referencji spowoduje wykonanie operacji odpowiedniej dla pełnego typu obiektu wywoływanego. Jeśli ma to miejsce w~trakcie działania programu, nazywa się to późnym wiązaniem lub wiązaniem dynamicznym.
	\item[Dziedziczenie] Umożliwia definiowanie specjalizowanych obiektów na podstawie ich ogólniejszych odpowiedników. Podczas definiowania obiektów specjalizowanych nie jest wymagana redefinicja całej funkcjonalności obiektu bazowego, ale tylko ta, której brakuje w~obiekcie ogólniejszym, lub ta, której sposób działania chcemy zmienić.
	
\end{description}

\subsection{.NET Framework 4.0}
Wydany w roku 2010 w wersji 4.0 framework platformy stanowi podstawę implementacyjną stworzonego w ramach niniejszej pracy oprogramowania. Aplikacja stworzona została przy pomocy najnowszej wersji, czyli 4.0 w zintegrowanym środowisku programistycznym Visual Studio 2010 Professional. 

W wersji 4.0 rozszerzono wiele funkcji dostępnych we wcześniejszych wersjach - m.in. wprowadzająć PLINQ, czyli równoległe wykonywanie zapytań LINQ (\english{Language-INtegrated Query}), które zostało wykorzystane w jednym z zaimplementowanych w ramach niniejszej pracy algorytmie. Poza tym w implementacji wykorzystano szereg wprowadzonych w 4. wersji usprawnień - m.in. nazywane parametry, czy też parametry opcjonalne w metodach. 

Ponieważ nie jest celem niniejszej pracy opisywanie poszczególnych wersji języka, czy też platformy, pominięte zostały szczegóły odnośnie poszczególnych wersji tejże platformy. Doskonałym źródłem informacji o zmianach w poszczególnych wersjach, a także przyczynach zachodzących zmian jest książka Jona Skeeta~\cite{cSharp:inDepthS}.

\section{CUDA}
W realizacji projektu wykorzystana została uniwersalna architektura procesorów wielordzeniowych, która szerzej została omówiona w rozdziale~\ref{chap:03}. Użyto NVIDIA CUDA w wersji 3.0, dostępnej do pobrania pod adresem \url{http://developer.nvidia.com/cuda-toolkit-30-downloads} (zweryfikowana dostępność w dniu \today). Językiem wybranym do implementacji metod optymalizujących algorytm został CUDA C, czyli rozszerzenie języka C.

\subsection{CUDA na platformie .NET}
Jak opisano w części~\ref{sec:cs} językiem, który wykorzystano do napisania aplikacji testowej był C\# uruchamiany na platformie .NET. W nawiązaniu do rozdziału~\ref{chap:03}, w którym napisano, że NVIDIA wypuściła rozszerzenie języka C (tzn. CUDA C), w którym pojawiły się instrukcje bezpośrednio związane z uniwersalną architektruą wielowątkową CUDA.

Istnieje możliwość użyciać mechanizmów współpracy kodu zarządzanego i natywnego, jednakże istnieją metody znacznie prostrzego wykorzystania możliwośći GPU do obliczeń. Można wykorzystać specjalną bibliotekę, stworzoną do umożliwienia programistom skorzystania z CUDA C wewnątrz programów zarządzanych.

Aby skorzystać z karty graficznej wewnątrz programu przygotowanego na platformę .NET użyta została biblioteka o nazwie CUDA.NET~\cite{cuda:net}. Sposób wykorzystania CUDA wewnątrz aplikacji na platformę .NET przedstawione są poniżej.

\begin{enumerate}
	\item Aplikacja oraz związana z nią logika biznesowa (np. przygotowanie i wyświetlanie danych) pisana jest w języku platformy .NET - w wypadku programu powiązanego z niniejszą pracą był to C\# w wersji $4.0$.
	\item Funkcje obliczeniowe przygotowywane są w języku CUDA C - jako pliki z rozszerzeniem \emph{.cu}. Jest to zgodnie z konwencją pisania programów na platformę CUDA~\cite{Cuda:PGuide}.
	\item Pliki z punktu poprzedniego powinny być skompilowane przy użyciu kompilatora przygotowanego przez firmę NVIDIA - w wyniku tej operacji powstaną pliki o rozszerzeniu \emph{.cubin}. Poniżej zaprezentowane zostało polecenie dla kompilatora wykorzystywane w budowaniu aplikacji, która stanowi podstawę niniejszej pracy.
		\begin{quote}
			\begin{verbatim}
				> nvcc *.cu --cubin
			\end{verbatim}
		\end{quote}
	\item W aplikacji głównej programu należy dodać referencje do biblioteki CUDA.NET, a następnie dopisać kod pośredniczący, w którym przygotowane zostaną dane, uruchomione zostaną funkcje obliczeniowe na karcie graficznej, a następnie odczytane zostaną wyniki tych obliczeń.
\end{enumerate}

Opisaną powyżej architekturę w sposób schematyczny zaprezentowano na rysunku~\ref{rys:arch_cuda_net}.

\begin{figure}[h]
\centering
\includegraphics{figures/04/arch_cuda_net.png}
\caption{Schematyczne przedstawienie architektury aplikacji wykorzystującej technologię CUDA przy pomocy biblioteki CUDA.NET}\label{rys:arch_cuda_net}
\end{figure}

\section{Środowisko programistyczne}

\subsection{Zarządzanie zasobami projektu}
git

\subsection{Budowanie projektu}
psake

\section{Testowanie}

testy jednostkowe podstawowych funkcji systemu

\subsection{Usprawnienie procesu testowania}

\subsubsection{Code Contracts}

\subsubsection{Pex}

\chapter{Algorytm równoległy\label{chap:alg_rownolegly}}

Niniejszy rozdział można traktować, jako uzupełnienie rozdziału~\ref{chap:teoria}, w którym to dokonano opisu podstaw teoretycznych wraz z dwoma wiodącymi algorytmami do odkrywania reguł asocjacyjnch. W ramach nieniejszej pracy opracowany został algorytm równoległy, opisany w sekcji~\ref{sec:apriori} tego rozdziału.

Wcześniej zaś przedstawione zostanie zastosowanie algorytmu Apriori (opisanego w części~\ref{apriori:section} pracy), w którym wykorzystane zostały funkcje dostarczone przez najnowsze wydanie frameworka .NET. 

W rozdziale tym wykorzystywane są oznaczenia wprowadzone w rozdziale~\ref{chap:teoria}.

\section{Algorytm ParallelApriori\label{sec:papriori}}
Pierwszym algorytmem wykorzystującym możliwości procesorów równoległych jest ParallelApriori - stworzony na potrzeby niniejszej pracy. 

Jak wspomniano już wcześniej, zagadnienie odkrywania reguł asocjacyjnych można podzielić na dwa etapy~\cite{Problem:Statement}:
\begin{enumerate}
	\item Odkrywanie zbiorów częstych, których wartość wsparcia jest wyższa od wartości $minsup$.
	\item Generowanie reguł asocjacyjnych na podstawie znalezionych zbiorów częstych.

	Na tym etapie możliwe jest tworzenie reguł, w których w zbiorze \emph{poprzedników} ($X$ z oznaczeń z definicji~\ref{regula:def}) jest wiele elementów oraz jeden w \emph{następniku} (zbiór $Y$ z definicji~\ref{regula:def})~\cite{Problem:Statement} lub dopuszczana jest możliwość wielu elementów również w następniku~\cite{Apriori:Main}. Niniejsza praca analizuje algorytmy generujące reguł, w któych oba zbiory mogą być zbiorami wieloelementowymi.
\end{enumerate}

\subsection{Generowanie zbiorów częstych}\label{papriori:gen}
Podobnie jak klasyczny algorytm Apriori~\cite{Apriori:Main}, ParallelApriori dokonuje analizy bazy danych $DB$, by w kolejnych iteracjach generowac rodziny coraz to liczniejszych zbiorów, będących zbiorami częstymi dla zadanej wartości $minsup$. Algorytm zaczyna od znalezienia wszystkich zbiorów jednoelementowych, które są zbiorami częstymi. Następnie w każdym kolejnym kroku generowane są zbiory częste na podstawie zbiorów wygenerowanych w kroku poprzednim. Proces ten jest kontynuowany do momentu aż nie zostaną znalezione żadne zbiory częste.

Algorytm generuje zbiory kandydatów jedynie na podstawie zbiorów częstych odkrytych w kroku poprzednim - co ważne generowanie ich odbywa się bez wielokrotnego przeglądania bazy danych transakcji. Każdy zbiór częsty zawierający $k$ elementów może być wygenerowany na podstawie połączenia dwóch zbiorów posiadających $k-1$ elementów, a na koniec kasując te zbiory, których jakikolwiek podzbiór nie jest częsty~\cite{Apriori:Main}.

Procedura \proc{Apriori Frequent Set Generaion} przedstawia pseudokod realizujący opisywany w tym rozdziale algorytm generowania zbiorów częstych.

\begin{codebox}
	\Procname{$\proc{Apriori Frequent Set Generaion}$}\label{apriori:listing}
	\li $\id{L_1} \gets \lbrace 1$-zbiory częste $\rbrace$
		\li \For $(k = 2; \id{L_{k-1}} \neq \emptyset; k++)$
		\li \Do
			 $\id{C_k} \gets pAprioriGen(\id{L_{k-1}})$
			\li \For \textbf{each} trasakcja $t \in \id{DB}$ \textbf{as parallel}
			\li \Do
					$C_t \gets subset(C_k, t)$
					\li \For \textbf{each} kandydat $c \in \id{C_t}$ \textbf{as parallel}
					\li \Do c.count++
					\End
				\End
			\li $L_k \gets \lbrace c \in C_k | c.count \geq minsup \rbrace$	
		\End
	\li Answer $\gets \bigcup_k L_k $
\end{codebox}

\subsubsection{Procedura pAprioriGen}

Procedura \id{pAprioriGen} reprezentuje proces twórzenia zbiorów $k-$ elementowych kandydatów na podstawie zbiorów wejściowych $(k-1)$-elementowych. 

Jak łatwo zauważyć wynikiem działania \proc{Parallel Join Step} są zbiory $k$-elementowe, które powstały na podstawie zbioru zbiorów wejściowych $L_{k-1}$, a ich zawartość różni się tylko jednym, ostatnim elementem. Ważnym faktem jest to, iż elementy w zbiorach są uporządkowane leksykograficzne, co znacząco ułatwia implementację tej procedury.

\begin{codebox}
	\Procname{$\proc{Parallel Join Step}$}
	\li \textbf{insert into} $C_k$
	\li \textbf{select} p.item$_1$, p.item$_2$, \dots, p.item$_{k-1}$, q.item$_{k-1}$ \textbf{as parallel}
	\li \textbf{from} $L_{k-1}$ p, $L_{k-1}$ q
	\li \textbf{where} p.item$_1 = $ q.item$_1$, \dots, p.item$_{k-2}$ = q.item$_{k-2}$, p.item$_{k-1}$ $<$ q.item$_{k-1}$
\end{codebox}

Warto zauważyć, że \proc{Parallel Join Step} jest ekwiwalentem rozszerzania zbioru $L_{k-1}$ każdym elementem zbioru elementów $I$, a następnie kasowania tych $(k-1)$-zbiorów otrzymanych przez usuwanie $(k-1)$ elementu, które nie są w $L_{k-1}$. 

Widać, że procedura \proc{Parallel Join Step} jest rozszerzeniem procedury \proc{Join Step} z części~\ref{sec:apriori} niniejszej pracy. Rozszerzona została ona o modyfikator \textbf{as parallel}, który opisany zostanie w kroku następnym.

W przypadku tego algorytmu zdecydowano się nie wprowadzać procedury \proc{Prune Step} obecnej w klasycznej implementacji algorytmu Apriori. Decyzja była podjęta na podstawie intuicji, że generowanie podzbiorów pewnego zbioru i porównywanie z innym zbiorem jest bardziej kosztowne niż sprawdzenie wartości $sup$ dla danego zbioru $k$-elementowego, będącego jednym z elementów wyniku działania procedury \id{pAprioriGen}.

\subsubsection{Kwantyfikator as parallel czyli Parallel LINQ}\label{sec:asparallel}
We wcześniejszych opisach procedur znajdowania zbiorów częstych wprowadzone został kwantyfikator \textbf{as parallel}, który jest głównym elementem opisywanego algorytmu ParallelApriori. 

Jak wspomniano w części~\ref{sec:linq} w implementacji algorytmu Apriori technologia LINQ była jedną z częściej wykorzystywanych na obiektach. W roku 2006 Joe Duffy zapowiedział na swoim blogu wprowadzenie nowej technologii, rozszerzającej możliwości LINQ~\cite{cs:helloplinq}. PLINQ (lub też Parallel LINQ) zostało włączone w skład .NET 4, pozwalając budować programy równoległe w bardziej abstrakcyjny sposób. 

Głównym założeniem PLINQ jest fakt, że biorąc wyrażenie LINQ, które wykonywane jest w stosunkowo długim czasie, można zapewnić szybsze jego wykonanie poprzez użycie wielu \emph{wątków} (\english{threads}), które wykorzystają możliwości wielu rdzeni na komputerze - z jak najmniejszą liczbą zmian w kodzie, jak to jest tylko możliwe. Jak w przypadku innych kwestii związanych z programowaniem równoległym - nie jest to aż tak proste podejście, jednakże nawet zwykła zmiana wyrażeń prowadzi zazwyczaj do przyspieszenia działania aplikacji.

Punktem startu dla przeobrażenia wyrażenia LINQ do jego wersji równoległej jest metoda 

\begin{verbatim}
AsParallel()
\end{verbatim}

Jej zastosowanie przekształca wyrażenie, które zostaje podzielone, a następnie będzie wykonywane w środowisku równoległym. Stąd też zastosowanie słowa kluczowego \textbf{as parallel} oznacza, że wyrażenie implementowane w przypadku klasycznego algorytmu Apriori (patrz część~\ref{apriori:section}) w LINQ zostało przekształcone w miarę możliwości do nowego, wprowadzonego w ostatniej wersji frameworka PLINQ.

\subsubsection{Generowanie reguł asocjacyjnych}
Zgodnie z zasadą działania algorytmu Apriori po zakończeniu pierwszego etapu algorytm ParallelApriori przystępuje do etapu drugiego, czyli do budowania reguł asocjacyjnych na podstawie odkrytych zbiorów. Podobnie, jak w~\cite{Apriori:Main} algorytm będący przedmiotem analizy niniejszej pracy, generuje wszystkie możliwe reguły asocjacyjne dla zadanego zbioru - zupełnie jak ma to miejsce w implementacji algorytmu opisanej w~\ref{sec:genrules}.

Aby wygenerować reguły, dla każdego zbioru częstego $l$ znajdowane są niepuste podzbiory - podzbiór taki oznaczony jest jako $a$. Dla takich oznaczeń wygenerowna zostanie reguła $a \Rightarrow (l-a)$, jeżeli spełniona jest nierówność $\frac{support(l)}{support(a)} \geq minconf$. Warto zauważyć, że dla każdego zbioru częstego generowane są wszystkie możliwe niepuste podzbiory - zapewnia to, że odkryte zostaną wszystkie możliwe reguły dla zadanego zestawu danych.

Istnieje możliwość poprawienia tego podejścia - podejście to zostało opisane we wcześniejszych rozdziałach. Nie zostało wzięte pod uwagę w implementacji ParallelApriori, by zachować podstawowe cechy algorytmu Apriori - by eksperymenty były możliwie wiarygodne.

Procedura~\proc{Parallel Generate Frequent Itemsets} prezentuje generowanie reguł asocjacyjnych na podstawie odkrytych $k$-zbiorów częstych $l_k$ będących elementami zbioru $L_k$ ($l_k \in L_k$).

\begin{codebox}
	\Procname{$\proc{Parallel Generate Frequent Itemsets}$}
		\li \For \textbf{each} zbiór częsty $l_k$, $k \geq 2$  \textbf{as parallel}
		\li \Do
			\textbf{call} pGenrules($l_k$, $l_k$)
			\End
		\End
\end{codebox}

W powyższym algorytmie wykorzystana została funkcja \proc{pGenrules}, która na podstawie dwóch zbiorów generuje reguły asocjacyjne. Zapis pseudokodu tej funkcji przedstawiony jest poniżej.

\begin{codebox}
	\Procname{$\proc{pGenrules}$($l_k$: $k$-zbiór częsty, $a_m$: $m$-zbiór częsty)}
		\li $\id{A} \gets \lbrace (m-1)$-zbiór $a_{m-1} | a_{m-1} \subset a_m \rbrace$
		\li \For $a_{m-1} \in A$ \textbf{as parallel}
			\li \Do
			$\id{conf} \gets \frac{support(l_k)}{support(a_{m-1})}$
			\li \If $\id{conf} \geq \id{minconf}$
				\li \Then
						\textbf{output} reguła $a_{m-1} \Rightarrow (l_k - a_{m-1})$ \\ ufność = $conf$ oraz wsparcie= $support(l_k)$
						\li \If $m-1 > 1$ 
							\li \Then
							\textbf{call} pGenrules($l_k$, $a_{m-1}$) \\ generowanie reguł podzbiorów zbioru $a_{m-1}$
						\End
				\End
			\End
		\End
\end{codebox}

Zupełnie, jak to miało miejsce w przypadku opisu generowania zbiorów częstych dla zadanego zbioru danych, można zauważyć stosowanie słowa kluczowego \textbf{as parallel}, którego znaczenie opisane zostało w części~\ref{sec:asparallel}.

\section{Algorytm CudaApriori\label{sec:capriori}}
\chapter{Eksperymenty\label{chap:eksperymenty}}

\chapter{Podsumowanie i~wnioski\label{chap:zakonczenie}}

Jak napisano we wprowadzeniu do~niniejszej pracy, informatzacja powinna realizować dwa podstawowe cele. Z~jednej strony powinna ona usprawniać pracę pojedynczego pracownika, poprzez automatyzację realizowanych przez niego rutynowych zadań i~obowiązków. Działania jednostki powinny być wykonywane szybciej i~bardziej niezawodnie, dzięki wykorzystaniu możliwości komputerów. Z~drugiej strony informatyzacja powinna również wpływać na~działania całych firm i~instytucji, poprzez wspomaganie procesu podejmowania decyzji przez kadrę zarządzającą przedsiębiorstwami (lub innymi jednostkami). Szybka analiza danych o~stanie firmy może umożliwić podejmowanie trafniejszych decyzji przez kadrę zarządzającą - ma to~duże znaczenie w~przypadku decyzji o~znaczeniu strategicznym dla~rozwoju przedsiębiorstwa.

W chwili obecnej praktycznie w~każdej przestrzeni ludzkiego życia znajduje się~komputer, co~wpłynęło na~produkcję olbrzymich ilości danych. W~większośći są one reprezentowane w~sposób umożliwiający ich łatwe składowanie i~przetwarzanie komputerowe przez aplikacje analityczne. Z~roku na~rok rozmiar danych produkowanych przez różnego rodzaju systemy komputerowe rośnie w~oszałamiającym tempie. Dane te poddane analizie mogą przynieść wymierne korzyści zarówno w~kwestiach finansowych, jak również poznawczych. Dzięki analizie zebranych w~przeszłości informacji możliwe jest lepsze dopasowanie planów w~przyszłości - na~tej podstawie planowane mogą być np. akcje marketingowe, czy~też~promocje w~supermarketach spożywczych. Wiedza uzyskana w~ten sposób może być wykorzystana w~bardzo różnorodnych sytuacjach, praktycznie w~każdym obszarze działalności firmy.

Proces odkrywania zależności pomiędzy zgromadzonymi danymi bez wykorzystania systemów komputerowych jest niezwykle skomplikowany i~wymaga do~realizacji dużo czasu. W~chwili obecnej analiza zgromadzonych danych przez człowieka jest praktycznie niemożliwa, bądź na~tyle powolna, że~wnioski nie~będą przydatne, bo~sytuacja ulegnie zmianie na~tyle, że~informacje z~analizy przestaną być aktualne. Z~tego też powodu tworzone są narzędzia umożliwiające odkrywanie prawidłowości w~dużych zbiorach danych, by człowiek na~tej podstawie mógł podejmować trafne i~szybkie decyzje oraz wyciągać odpowiednie wnioski.

Eksploracja danych jest działem informatyki, który~zajmuje się~poszukiwaniem ukrytych dla~człowieka prawidłowości i~reguł w~danych. Jest to~jeden z~etapów procesu \termdef{odkrywania wiedzy z~baz danych}. Jak~wspomniano we wprowadzeniu, eksploracja danych to~proces poszukiwania wiedzy w~postaci nowych, użytecznych, poprawnych i~zrozumiałych wzorców w~bardzo dużych wolumenach danych~\cite{DataMiningStart}. Możliwości stosowania technik eksploracji danych w~praktyce, wymagają efektywnych metod przeszukiwania ogromnych plików lub~baz danych. Warto przy tym wspomnieć, że~tego typu technologie nie~są w~chwili obecnej dobrze zintegrowane z~systemami zarządzania bazami danych.

Reguły asocjacyjne są jednym z~najczęściej używanych modeli wiedzy w~eksploracji danych. Jak~przedstawiono w~rozdziale~\ref{chap:teoria} reguła asocjacyjna ma postać $X \Rightarrow Y$, gdzie~$X$ oraz~$Y$ są wzajemnie rozłącznymi zbiorami elementów. Przykładem reguły, która~mogła zostać odkryta w~bazie danych sklepu komputerowego, może być reguła postaci $komputer \land myszka \Rightarrow monitor$. Prezentuje ona fakt, że~klienci kupujący komputer oraz myszke, z~dużym prawdopodobieństwem kupią również monitor. Problem odkrywania reguł asocjacyjnych wraz z~algorytmem Apriori, który~jest podstawą wielu algorytmów znajdujących reguły asocjacyjne po raz pierwszy został sformułowany w~\cite{Problem:Statement}. Algorytm ten~następnie rozszerzono w~pracy~\cite{AssRulesStrt} o~bardziej uniwersalne sposoby generowania reguł. Innym rodzajem algorytmu odkrywającego reguły asocjacyjne jest FPGrowth, który~opisany jest w~części~\ref{sec:fpgrowth} niniejszej pracy.

W ostatnich latach pojawiły się~nowe możliwości wykorzystania współczesnych komputerów. W~roku 2007 firma NVIDIA udostępniła programistom uniwersalną architekturę obliczeniową CUDA (\english{Compute Unified Device Architecture}), który~umożliwia wykorzystanie mocy obliczeniowej \termdef{procesorów graficznych} (\acronym{GPU}, \english{Graphics Processing Unit}), bądź innych procesorów wielordzeniowych, do~rozwiązywania ogólnych problemów obliczeniowych w~sposób znacząco wydajniejszy niż w~przypadku tradycyjnych, sekwencyjnych procesorów~\cite{cuda:zone}. W~ramach tworzenia projektu aplikacji porównującej różne algorytmy odkrywające reguły wykorzystana została technologia CUDA, którą~najpierw autor musiał poznać i~w~odpowiedni sposób wykorzystać w~pracy nad implementacją.

Jak do~tej pory bardzo małe jest zainteresowanie wykorzystaniem tej~technologii w~procesie odkrywania wiedzy, a~w~szczególności znajdowania reguł asocjacyjnych. Wyniki przeprowadzonych w ramach tej pracy eksperymentów pozwalają na wyciągnięcie wniosków, że~algorytm wykorzystujący możliwości procesorów wielordzeniowych, po~w~szczególności GPU, będzie wyraźnie szybszy od~klasycznych algorytmów eksploracji danych zaimplementowany na~tradycyjnych procesorach. 

W niniejszej pracy opracowany został algorytm równoległy oparty na~Apriori, który~wykorzystuje możliwości wpółczesnych kart graficznych poprzez wykorzystanie technologii CUDA. Eksperymenty wykazały, że~algorytm ten~jest znacząco szybszy od~innych implemntacji algorytmów z~tej klasy, co~pozwala na~przetworzenie większych zbiorów danych w~tych samych czasach. Z~przeprowadzonych badań jasno wynika również,~że algorytm FPGrowth jest algorytmem wydajniejszym niż~algorytm Apriori, co~może stanowić wskazówkę dla~dalszych prac nad~tematem zrównoleglania algorytmów odkrywających reguły acocjacyjne.

W rozdziale~\ref{chap:eksperymenty} poświęconym eksperymentom zaprezentowano wyniki dla~trzech algorytmów klasy Apriori oraz klasycznego algorytmu FPGrowth. Wydajność tego drugiego znacząco przekraczała wydajności pozostałych algorytmów. W~przyszłości celem jest przyspieszenie algorytmu FPGrowth poprzez zastosowanie technologii CUDA do~zrównoleglenia obliczeń. Dodatkowo potrzebne byłoby przetestowanie algorytmów na~większych zbiorach danych, gdzie odpowiedzi FPGrowth byłyby zależne od~liczby transakcji.

% All appendices and extra material, if you have any.
%\cleardoublepage\appendix%
%\input{0a-zalacznik.tex}
%\input{0b-pisanie-w-latexu.tex}

% Bibliography (books, articles) starts here.
\bibliographystyle{plalpha}{\raggedright\sloppy\small\bibliography{bibliography}}

% Colophon is a place where you should let others know about copyrights etc.
\ppcolophon

\end{document}
